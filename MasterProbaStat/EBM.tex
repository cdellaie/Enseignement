\section{Modèles à énergie}

Soit $(\Omega,\mathcal{A},\lambda)$ un espace mesuré, représentant l'espace des configurations d'un système. On suppose de plus que l'on dispose d'une fonction 
\[ H : \Omega \rightarrow \R \]
que l'on appellera énergie.\\

Soit $\beta>0$. La loi de probabilité qui gouverne le système à la température $\frac{1}{\beta}$ est donnée par :
\[P_\beta (X\in A) = \int_A \frac{e^{-\beta H(\sigma)}{Z_\beta} d\lambda(\sigma),\]
où $Z_\beta = \int_\Omega e^{-\beta H(\sigma)} d\lambda(\sigma)$ est la constante de normalisation du système.

\begin{enumerate}
\item Décrire le modèle statistique.
\item On se place dans le cas $(\R^q\times\R^p, \mathcal{B},Leb_{p+q})$, et $H(q,p) = \frac{1}{2} p^2 + V(q) $.
\item Matrices aléatoires.
\item Réseau de neurones.
\end{enumerate}
