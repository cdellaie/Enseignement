\section{Intervalles de confiance}
\begin{enumerate}
\item 
Soient $\mu\in \R$, $\sigma>0$ et $\alpha\in (0,1)$.
On observe un $n$-échantillon $\underline x=(X_1,...,X_n)$ de variables iid de loi normale $\mathcal N(\mu ,\sigma^2)$.
\begin{enumerate}
\item Donner un intervalle de confiance de niveau $\alpha$ pour $\mu$, si $\sigma^2$ est connu, puis si $\sigma^2$ est inconnu.
\item Donner un intervalle de confiance de niveau $\alpha$ pour $\sigma^2$, si $\mu$ est connu, puis si $\mu$ est inconnu.
\end{enumerate}

\item On observe un $n$-échantillon $\underline Y=(Y_1,...,Y_n)$ de variables iid suivant une loi de Bernoulli de paramètre $0<p<1$ inconnu. Donner un intervalle de confiance pour $p$ au niveau $\alpha$.

\end{enumerate}

\section{Estimateur du maximum de vraisemblance}
On observe un $n$-échantillon $\underline x=(X_1,...,X_n)$ de variables iid suivant une loi $\mathbb{P}_\theta$ de densité :
\[f_\theta (x ) = \frac{2}{\sqrt{\pi}\theta^{3/2}}x^2 e^{-\frac{x^2}{\theta}}, \text{ pour }\theta >0. \]

\begin{enumerate}
\item  Décrire le modèle statistique engendré par $\underline x$.
\item Calculer un estimateur du maximum de vraisemblance, que l'on notera $\hat \theta$.
\item Examiner les qualités suivantes de $\hat \theta$ : efficacité, biais et convergence.
\end{enumerate}

On observe un $n$-échantillon $\underline x=(X_1,...,X_n)$ de variables iid suivant une loi exponentielle de paramètre $\lambda >0$, de densité
\[g_\lambda (x)=\lambda e^{-\lambda x}1_{x\geq 0}.\]
\begin{enumerate}
 \item Décrire le modèle statistique engendré par $\underline x$.
\item Calculer la vraismeblance du modèle.
\item Donner un estimateur du maximum de vraisemblance pour $\lambda$.
\end{enumerate}


