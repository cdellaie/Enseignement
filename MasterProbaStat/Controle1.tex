\textbf{Nom :}\hfill \textbf{06/02/2017}\\
\textbf{Prénom :}

\section*{Contrôle de connaissances}

\begin{enumerate}
\item Enoncer la loi des grands nombres.
\item Enoncer le théorème de la limite centrale.
\item Vous êtes contacté-e par un institut pour effectuer un sondage sur une population de $n$ individus, $n$ étant supposé très petit devant la taille de la population totale $N_{tot}$. Le sondage concerne l'intention de vote des participants pour une élection à deux candidats, $A$ et $B$. Décrire un modèle statistique que vous utiliseriez pour répondre à la problématique. Comment et quand déciderez-vous quel candidat a le plus de chance de gagner ?
\item On se donne des variables aléatoires $X_1$,...,$X_n$ i.i.d. d'espérance $\mu\in\mathbb R$ et de variance $\sigma^2>0$. On pose $\overline X_n = \frac{1}{n}\sum_{i=1}^n X_i$. Calculer 
\[E[\overline X_n]\text{ et } E[\frac{1}{n}\sum_{i=1}^n (X_i-\overline X_n)^2].\]
\end{enumerate}
