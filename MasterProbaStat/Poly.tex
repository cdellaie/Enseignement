
\documentclass[a4paper]{report}

\usepackage[frenchb]{babel}
\usepackage{amsfonts}
\usepackage{amsmath}
%\usepackage[T1]{fontenc}
\usepackage[utf8]{inputenc}
\usepackage{amsthm}
\usepackage{graphicx}
\usepackage{tikz}
\usepackage{hyperref}
\usepackage{amssymb}
\usepackage{tikz-cd}

\usepackage{listings}
\lstdefinestyle{Scilab}{%
  language=Scilab,%
  breaklines=true,%
  frame=l,%
  xleftmargin=\parindent,%
  basicstyle=\ttfamily\small,%
  keywordstyle=\bfseries\color{green!40!black!80},%
  showstringspaces=false,%
  commentstyle=\itshape\color{purple!70},%
  identifierstyle=\color{blue!80},%
  stringstyle=\color{red!80},%
  directivestyle=\color{orange!90!black!80},%
  % otherkeywords={},%
  escapeinside={<latex>}{</latex>},%
}
\lstset{style=Scilab}

\hypersetup{                    % parametrage des hyperliens
    colorlinks=true,                % colorise les liens
    breaklinks=true,                % permet les retours à la ligne pour les liens trop longs
    urlcolor= blue,                 % couleur des hyperliens
    linkcolor= blue,                % couleur des liens internes aux documents (index, figures, tableaux, equations,...)
    citecolor= cyan               % couleur des liens vers les references bibliographiques
    }

\theoremstyle{definition}
\newtheorem{definition}{Définition}
\newtheorem{thm}{Théorème}
\newtheorem{ex}{Exercice}
\newtheorem{lem}{Lemme}
\newtheorem{dem}{Preuve}
\newtheorem{prop}{Proposition}
\newtheorem{cor}{Corollaire}
\newtheorem{conj}{Conjecture}
\newtheorem{Res}{Résultat}

\newcommand{\N}{\mathbb N}
\newcommand{\Z}{\mathbb Z}
\newcommand{\R}{\mathbb R}
\newcommand{\C}{\mathbb C}
\newcommand{\Hil}{\mathcal H}
\newcommand{\Mn}{\mathcal M _n (\mathbb C)}
\newcommand{\K}{\mathbb K}
\newcommand{\B}{\mathbb B}
\newcommand{\Cat}{\mathbb B / \mathbb K}

\setlength\parindent{0pt}

\usepackage{geometry}
\geometry{hmargin=2.5cm,vmargin=1.5cm}

\title{M1 Statistiques et Séries Chronologiques, Probabilités et processus aléatoires  \\ Université de Lorraine \\ ~ \\
%\textbf{Régression Linéaire Multiple et Initiation aux tests statistiques}}
\textbf{Enoncés des exercices de TP et TD}}

%\title{TP Statistiques et Séries Chronologiques  \\ Université de Lorraine \\ ~ \\
%\textbf{Partiel du $27$ mars $2015$}}

\date{} %la date
\author{ Clément Dell'Aiera }

\begin{document}  

\maketitle
\tableofcontents

\part{Statistiques et séries chronologiques}
%régression
\chapter{Régression linéaire}
\section{La régression linéaire sous $R$}

Ce numéro rappelle les notions nécessaires à l'interprétation d'une sortie $R$ de la fonction \textit{lm}. Dans toute la suite, \textit{regression} désigne un objet de type \textit{lm} que l'on a appellé grâce à $lm(Y\sim X_1+...+X_n, data=...)$. Si l'on note $X_{-i}$, le moins signifie que l'on calcule la quantité $X$ sans tenir compte de l'observation $i$.\\

Dans le modèle de régression \[Y=X\beta + \epsilon ,\]
la commande \textit{summary(regression)}, où \textit{regression} est un objet de la classe $lm$, renvoie plusieurs tableaux. 

On note $\hat y_j =  \sum h_{ij}x_j$ , i.e. \[h_{ij}= \frac{1}{n}+\sum \frac{(x_i-\hat x)(x_j-\hat x)}{\sum (x_j-\hat x)^2}.\]

Le premier tableau,\textit{Residuals}, est destiné à donner une idée de la répartition des résidus en affichant les quantiles.  Je vous recommande d'afficher tout de même les résidus, et d'observer leur distribution. Mieux, vous pouvez utiliser les résidus studentisés. A priori, bien que tous soient centrés, les résidus n'ont pas même variance (même sous hypothèse d'homoscédasticité !) : $Var[\epsilon_j]=\sigma^2(1-h_{jj})$. Pour les rendre comparables, on pourrait les réduire, mais si l'on remplace la variance par la variance estimée, celle-ci dépend de l'information contenue dans $x_j$, ce qui empêche une quantification de l'effet que $x_j$ a sur les coeffcients de la régression. Pour palier à ce problème, on introduit 
\[\hat \sigma_{-j}^2 = \frac{1}{n-3}[(n-2)\hat \sigma^2-\frac{\epsilon_j}{1-h_{jj}}]\]
qui n'est rien d'autre que la variance estimé sur le modèle où l'on a supprimé l'observation $x_j$. Les résidus studentisés sont définis par $T_j = \frac{\epsilon_j}{\sigma_{-j}(1-h_{jj})}\sim T(n-3)$ et suivent une loi de Student à $n-3$ degré de liberté sous des hypothèses raisonnables. Pour détecter une anomalie dans les données, on peut vérifier que les résidus studentisés se répartissent de manière uniforme sur l'intervalle $[-2;2]$ (sous hypothèse d'homoscédasticité). Repérer des formes suspectes est un moyen facile pour repérer les valeurs aberrantes. On peut par exemple taper : \\
\textit{qqnorm(studres(regression)); qqline(studres(regression))}.\\
Une autre méthode pour détecter les valeurs aberrantes : utiliser la distance de Cook. Elle est définie par 
\[D_j= \frac{\sum_j (\hat y_{-i,j}-\hat y_j)^2}{2\hat \sigma^2},\]
et mesure l'influence d'une observation sur l'ensemble des prévisions (qu'on veut petite!). Encore une règle du pouce : si $D_j>1$, on enlève l'observation $j$. Vous pouvez le faire automatiquement en tapant \textit{plot(regression,which = 4)}.

Le deuxième est nommé \textit{Coefficients} :
\[\begin{array}{|c|c|c|c|c|}
	\hline
	& \text{Estimate} & \text{Std. Error} 	 & \text{t-value} & Pr(>|t|) \\
	\hline
 \beta_j & \hat \beta_j 	& \hat \sigma_j 	 & \hat t_j =\frac{\hat \beta_j}{\hat \sigma_j} & p\text{-value}\\
\hline
\end{array}\]
\\
Les trois premières colonnes s'expliquent elles-mêmes, mais à quoi servent les $2$ dernières ? A effectuer un test de significativité. Plus précisément, on sait que $\hat t_j$ suit une loi de Student à $N-k$ degrés de liberté sous l'hypothèse $H_0 | \beta_j = 0$ contre $H_1 | \beta_j \neq 0$. La quatrième colonne donne donc la valeur de cette statistique, et la dernière sa $p$-value, définie comme la valeur seuil de confiance $\alpha$ qui fait basculer le test. (Rappelez vous que, mécaniquement, si $\alpha$ diminue assez, on finit par accepter $H_0$.) Une règle appliquée par les statisticiens est la suivante :\\

\begin{center}
\begin{tabular}{|c|c|}
\hline
$p<0.01$ & suspicion très forte contre $H_0$ \\
\hline
$0.01-0.05$ & suspicion forte contre $H_0$ \\
\hline
$0.05-0.1$ & suspicion faible contre $H_0$ \\
\hline
$>0.1$ & peu ou pas de suspicion contre $H_0$\\
\hline
\end{tabular} 
\end{center}


Donc, si $Pr(>|t|)$ est petit, on rejette l'hypothèse $\beta_j = 0$, ce qui signifie que le coefficient est significatif. $R$ ajoute même des petites étoiles à côté des coefficients les plus significatifs.\\

Reste encore à observer plusieurs indicateurs. L'erreur \textit{Residual standard error} est calculée comme un estimateur de $\sigma$ sous l'hypothèse de matrice variance-covariance égale à $\sigma^2 I_n$. Il nous reste encore le $R^2$ et le $R^2$ ajusté. Rappelons que le coefficient $R^2$ peut s'interpréter comme le cosinus de l'angle entre le vecteur des observations $Y_j$ et son projeté pour la norme $\mathcal L^2$ sur l'espace linéaire engendré par les observations $X_j$, soit
\[R^2 = \frac{\sum (\hat y_j-\overline y)^2}{\sum ( y_j-\overline y)^2}.\]
Cet indicateur à des valeurs comprises entre $0$ et $1$, la proximité avec $1$ indiquant une bonne adéquation du modèle aux données. Toutefois, son interprétation est sujette à caution : sa valeur augmente mécaniquement avec l'ajout de variables explicatives. En particulier, pour comparer la qualité de deux modèles au nombre de variables explicatives distinct, on lui préférera le $R^2$ ajusté, qui prend en compte ce nombre noté $k$ :
\[RR^2 = 1-(1-R^2)\frac{n-1}{n-k-1}.\] 

\section*{Exercice 1}
\begin{enumerate}
\item Que font les commandes $pnorm$, $qnorm$ , $rnorm$, $dnorm$ ? Utilisez l'aide de \textit{R}.
\item Tracer la fonction de densité de la loi normale. Faites varier les paramètres et afficher les différentes courbes sur le même graphique. 
\item Simuler 2 vecteurs $X$ et $Y$ contenant chacun $N=100$ variables indépendantes identiquement distribuées suivant une loi normale $\mathcal N (0,1)$.
\item Afficher les points de coordonnées $(X[j],Y[j])$ dans le plan, pour $j$ allant de $1$ à $100$.
\item Tracer la fonction de répartition empirique des $X[j]$.
\item Soit $\mathcal E$ une v.a. de loi exponentielle de paramètre $1$, et $U$ une v.a. suivant une loi uniforme sur $[0,2\pi]$. On pose 
\[(X,Y)=(\sqrt{\mathcal E}\cos (U), \sqrt{\mathcal E} \sin (U)).\]
Quelle est la loi du couple $(X,Y)$ ? (Vous pouvez le prouvez, ou observer grâce à \textit{R} ce qu'il se passe en simulant ces variables et en les traçant.)
\end{enumerate}

\section*{Exercice 2}

\begin{enumerate}
\item Une table est déjà en mémoire dans \textit{R} : la table \textit{stackloss}. Analyser la rapidement.
\item Tracer \textit{stack.loss} en fonction de \textit{Air.Flow}. Qu'en pensez-vous ?
\item Effectuer la régression linéaire de \textit{stack.loss} en fonction des autres variables. Quelles sont celles qui sont significatives ?
\end{enumerate}

\section*{Exercice 3}

Le fichier \textit{ozone.dta} contient les variables suivantes, pour une série de journées (qui sont ici nos individus) :
\begin{itemize}
\item l’identifiant de la journée,
\item le maximum d’ozone (variable maxO3)
\item l’heure à laquelle le maximum d’ozone a été obtenu (heure),
\item les températures à 6h, 9h, 12h, 15h, 18h (resp. T6 à T18)
\item la nébulosité à 6h, 9h, 12h, 15h, 18h (resp. Ne6 à Ne18)
\item la projection du vent sur l’axe est-ouest à 12h (Vx),
\item le maximum d’ozone de la veille (maxO3v).
\end{itemize}
Le but est de modéliser la valeur des pics d’ozone en fonction de grandeurs physiques facilement mesurables (température, heure, nébulosité, vent) afin d’avoir des approximations de la qualité de l’air faciles
et rapides à obtenir.\\


\begin{enumerate}
\item Importer la table, et afficher un résumé de ce qu'elle contient.
\item Tracer \textit{maxO3} en fonction de \textit{T12}, puis effectuer un régression linéaire. Ajouter la droite de régression sur le graphique. Soignez la présentation.
\item Afficher les résultats de la régression.
\item Extraire les résidus et tracer leur densité estimée.
\item Effectuer la régression de \textit{maxO3} sur toutes les variables, et supprimer récursivement celles qui ne sont pas significatives, jusqu'à ce qu'elles le soient toutes.
\end{enumerate}

\section*{Exercice 4}
\begin{enumerate}
\item Sur le site \textit{data.gouv.fr}, vous pourrez trouver des tables de données publiques en libre accès. Choisissez un thème qui vous intéresse, puis une table en conséquence. Télécharger là.
\item Les tables sont souvent au format $.xls$ : vous aurez besoin d'installer un package pour pouvoir les lire. La commande pour ce faire est \textit{install.packages("nom du package")}. Installer le package \textit{gdata}.
\item Analyser votre table.
\end{enumerate}

\section*{Exercice 5}

\begin{enumerate}
\item Télécharger et installer le package \textit{ISwR} (Introductory Statistics with R).
\item Utiliser la commande \textit{summary} pour analyser rapidement la table \textit{bp.obese}. L'échantillon provient d'un échantillon de population mexicaine en Californie, et la table décrit $3$ variables : le sexe (femme = $1$, homme =$0$), le ratio d'obésité (\textit{obese}) et la pression sanguine systolique en $mm$ de mercure (\textit{bp}).
\item Représenter les données dans un graphe, en utilisant des symboles différents pour les hommes et les femmes.
\item Expliquer la pression sanguine en fonction du ration d'obésité, puis du ratio d'obésité et du sexe.
\item Tracer sur un même graphe les courbes correspondant aux régressions dans les 2 modèles. Soignez la présentation (couleurs différentes, légende,...)
\end{enumerate}

\section*{Exercice 6}
\begin{enumerate}
\item Télécharger la librairie \textit{MASS}.
\item Analyser rapidement la table \textit{cats} et afficher les variables les unes en fonctions des autres par paires.
\item Effectuer une régression linéaire selon le modèle $Hwt \sim Bwt * Sex $. Cela apporte-t-il quelque-chose par rapport au modèle $Hwt \sim Bwt + Sex $ ? 
\item Visualiser les composantes de votre régression.
\item En extraire les prédiction, les coefficients, les résidus, lés résidus studentisés, et la formule du modèle.
\item Tracer le \textit{qqplot} des résidus studentisés ainsi que la première bissectrice.
\item Tracer le graphe des résidus contre les prédictions.
\item Tracer le graphe des distance de Cook.
\item Observer les attributs que vous donne \textit{summary}
\item Afficher le $R^2$ ajusté de la régression, le nombre de degrés de liberté résiduels, la matrice de variance-covariance des paramètres estimés. 
\end{enumerate}



%Tests
\chapter{Tests statistiques}
\section{Théorie des tests}

\subsection{Exemples}
\begin{enumerate}
\item Une entreprise vend des biens dont elle assure que la durée de vie dépasse $10000$ heures. Vous êtes engagés pour vérifier la qualité d'iceux. Vous disposez d'un échantillon de $30$ de ces biens. La durée de vie moyenne calculée sur l'échantillon vaut $9900$ heures, on suppose que l'écart-type est connu et vaut $120$ heures. Pouvez-vous rejeter leur assertion à à un niveau de confiance de $0.05\%$.\\
Répondez à la même question si l'écart-type n'est plus connu, et que l'écart-type obtenu sur l'échantillon vaut $125$ heures.\\
Calculez la puissance du test, c'est-à-dire la probabilité de l'erreur de seconde espèce.

\item Une firme agroalimentaire assure qu'un cookie qu'elle produit ne contient pas plus de $2$ grammes d'un certain composé (graisse, colorant,...). Vous achetez un paquet, contentant $35$ cookiees, et mesurez une teneur moyenne de $2.1$ grammes. En supposant que l'écart-type de l'échantillon est de $0.25$, pouvez-vous incriminez la firme à un niveau de confiance de $0.05\%$. Même question si l'on ne connaît que l'écart-type empirique de $0.3$. Calculez la puissance du test.

\item Lors des dernières élections, les médias affirment qu'au moins $60\%$ des citoyens ont voté. Vous interrogez $148$ citoyens de façon à obtenir un échantillon représentatif de la population (vous êtes statisticien après tout). Vous obtenez que $85$ des personnes interrogées ont voté. A $0.05\%$, votre test concorde-t-il  avec l'affirmation des médias ?
\end{enumerate}
\subsection{Analyse of Variance ou procédure ANOVA}
\begin{enumerate}
\item Une institution de santé publique veut comparer l'effet de trois traitement contre la grippe. Pour cela, $18$ hopitaux sont choisis de façon aléatoire, répartis par groupes de $6$, chacun appliquant un et un seul des trois traitement. Voici le nombre de personnes guéries au bout d'une semaine de traitement : \\

\begin{center}
\begin{tabular}{|c|c|c|}
\hline
Trait. $1$ & Trait. $2$ & Trait. $3$ \\
\hline
22 & 52 & 16 \\
42 & 33 & 24 \\
44 & 8 & 19 \\
52 & 47 & 18 \\
45 & 43 & 34 \\
37 & 32 & 39 \\
\hline
\end{tabular}
\end{center}

\begin{enumerate}
\item Une méthode pour rentrer les données dans \textit{R} : les taper dans un fichier \textit{.txt} puis utiliser \textit{read.table}  pour créer un objet de la classe \textit{data frame}. (Faîtes-le)
\item Utiliser un test ANOVA pour répondre à la problématique de l'institution.
\item Un pays frontalier, lui aussi touché par l'épidémie, décide de répliquer l'expérience avec le même nombre d'hôpitaux et les mêmes traitements. Chaque hôpital est selectionné de façon aléatoire, et dois appliquer les trois traitements pendant trois semaines, chaque traitement pendant une semaine, l'ordre des traitements étant lui aussi aléatoire. Voici le résultat :

\begin{center}
\begin{tabular}{|c|c|c|}
\hline
Trait. $1$ & Trait. $2$ & Trait. $3$ \\
\hline
22 & 52 & 16 \\
42 & 33 & 24 \\
44 & 8 & 19 \\
52 & 47 & 18 \\
45 & 43 & 34 \\
37 & 32 & 39 \\
\hline
\end{tabular}
\end{center}
\end{enumerate}

\end{enumerate}




\section{Régression sur variables qualitatives}

Le but de cet exercice est d'expliquer la concentration en ozone \textit{O3} en fonction de la température \textit{T12} et de la direction du vent \textit{vent} dans la table \textit{ozone.txt}.\\

\begin{enumerate}
\item Télécharger la table, et effectuer des régressions selon les différents modèles.
\item Tester l'égalité des pentes.
\item Tester l'égalité des ordonnées à l'origine.
\item Analyser les résidus.
\end{enumerate}

\subsection{ANOVA à $1$ facteur}

Nous souhaitons modéliser la concentration en ozone en fonction de la direction du vent.

\begin{enumerate}
\item Tracer une boîte à moustaches de la variable $O3$ par rapport aux quatres modalités de la variable \textit{vent}. Le vent semble-t-il avoir une influence sur la concentration en ozone ?
\item On se place dans un modèle d'analyse de la variance à un facteur
\[y_{ij}=\mu + \alpha_j+\epsilon_{ij}\]
\begin{enumerate}
\item Effectuer la regression linéaire de \textit{O3} sur \textit{vent} sous la contrainte $\mu = 0$.
\item Effectuer la regression linéaire de \textit{O3} sur \textit{vent} sous la contrainte $\alpha_1 = 0$.
\item Effectuer la regression linéaire de \textit{O3} sur \textit{vent} sous la contrainte $\sum n_i \alpha_i = 0$.
\item Effectuer la regression linéaire de \textit{O3} sur \textit{vent} sous la contrainte $\sum n_i \alpha_i = 0$.
\end{enumerate}
\item Analyser les résidus afin de constater que l'hypothèse d'homoscédasticité est vérifiée. Pour cela, tracer un boxplot des résidus en fonction de \textit{vent}, les résidus en fonction de $\hat{O3}$, leurs quantiles théoriques ainsi que la distribution des résidus par modalité de \textit{vent}.
\end{enumerate}

\subsection{ANOVA à $2$ facteurs}

Nous voulons maintenant modéliser la concentration en ozone par le vent et la nébulosité, variable à $2$ modalités : SOLEIL et NUAGEUX.\\

\begin{enumerate}
\item Procéder à un examen graphique qui puisse déterminer si l'interaction des facteurs influe sur la variable à expliquer. (voir ce qu'est un \textit{profil})
\item On suppose la gaussianité des résidus. 
\begin{enumerate}
\item Tester le modèle avec interaction : \textbf{mod1}.
\item Tester le modèle sans interaction : \textbf{mod2}.
\item Tester le modèle sans effet du facteur \textit{nebulosité} : \textbf{mod3}.
\end{enumerate}
 \item Grâce à la commande ANOVA de R, effectuer des analyses de la variance entre les modèles \textbf{mod1}, \textbf{mod2} et \textbf{mod3}.
\item Répondez à la problématique.
\end{enumerate}


%\section{Décryptage par analyse fréquentielle}

%\begin{enumerate}
%\item Créer une fonction qui, si l'on lui fournit un texte et une permutation des lettres de l'alphabet en entrée, retourne le texte crypté à l'aide de la permutation.
%\item Créer une fonction qui effectue l'analyse fréquentielle d'un texte.
%\item Arrivez-vous à décrypter le message suivant ? Mettre un message. 
%\end{enumerate}{enumerate}

\chapter{Partiel}
\section{Follicule de rose}

Cet exercice est fortement inspiré d'un exercice du livre \textit{Initiation à la statistique avec R} de F. Bertrand et M. Maumy-Bertrand aux édition Dunod. Vous renderez une copie sur laquelle chaque question sera détaillée, et vous enverrez votre fichier $.R$ à mon adresse mail ainsi que tous vos graphiques. Votre fichier $.R$ devra être nommé \textit{NomPrénom.R}. Lorsque l'on demande de donner des conclusions, cela signifie que l'on s'attend à une \textbf{phrase rédigée} sur la copie, ainsi que du code \textbf{commenté} dans le fichier $.R$. De plus, les graphiques et autres tableaux ne peuvent qu'être bénéfiques à la compréhension, et donc à la notation, de votre travail.\\

Cet énoncé s'intéresse à quantifier l'effet de la taille des follicules de roses sur leurs masse, ainsi que l'effet possible de l'espèce. Dans la première partie, on ne s'intéresse qu'à la relation entre taille et masse, la seconde essaie d'incorporer l'espèce (qui est une variable qualitative).\\

\begin{enumerate}
\item Télécharger le package \textit{BioStatR} et importer la librairie ainsi que la table \textit{Mesures}. Décrire rapidement la table \textit{Mesures}. (Nombre de variables, principales caractéristiques des variables,etc.)
\item Tracer le nuage de points de la taille en fonction de la masse des follicules de roses.
\item Pouvez-vous soupçonner une relation linéaire entre ces deux variables ?
\item Effectuer la régression linéaire de la taille sur la masse. Tracer la droite de régression sur le graphique. Soignez la présentation (titre, couleurs,etc). 
\item Donner un intervalle de confiance pour les coefficients du modèle que vous obtenez. Comment avez-vous trouvé ces informations ? En bonus, vous pouvez tracer la zone de confiance pour la droite de régression sur votre graphique.
\item Donnez une prédiction plausible de la taille d'un follicule si sa masse est de $42$ grammes.
\item Comment appellez-vous l'écart entre la valeur observée d'une observation et celle prédit par le modèle ?
\item La droite des moindres carrés passe-t-elle par le point moyen $(\overline x_n, \overline y_n)$ ? Si oui, démontrer-le.
\item Donner le coefficient de détermination $R^2$. A quoi sert ce coefficient ? Quelle est la différence entre le $R^2$ et le $R^2$ ajusté ?
\item Donner la part de variance expliquée et totale.
\item Donnez la valeur de l'estimation de $\sigma^2$.
\item Effectuer les deux tests suivants au risque $0.05\%$ :
\begin{itemize}
\item $H_0 \ | \ \beta_0 = 0$ contre $H_1 \ | \ \beta_0 \neq 0$.
\item $H_0 \ | \ \beta_1 = 0$ contre $H_1 \ | \ \beta_1 \neq 0$.
\end{itemize}
\end{enumerate}

Cette partie vise à quantifier l'effet de l'espèce sur la taille du follicule. 

\begin{enumerate}
\item Effectuer une régression croisée qui tient compte de la variable \textit{espece}. Donner une synthèse rapide du résultat.
\item Comment pourriez-vous tester si l'espèce à un effet sur la taille ? 
\item Proposer une méthode, le mettre en oeuvre et donnez vos conclusions.
\end{enumerate}
%Projet
\section{Problème du voyageur de commerce et algorithme du recuit simulé}
D'après le livre de Michel Benaïm et Nicole El Karoui, \textit{Promenade aléatoire, Chaîne de Markov et simulations, martingales et stratégies}, exemple $3.1.8$. \cite{Benaim_ElKaroui}\\

La méthode du recuit simulé est un algorithme d'optimisation proche de celui de Metropolis, et consiste à se promener aléatoirement sur l'espace (fini) des états d'un système selon une loi construite de façon à ce que la promenade converge vers un état qui minimise une certaine fonctionnelle.\\

On se donne une fonction $h :]0;\infty[\rightarrow ]0;1]$ telle que 
\[h(x)=xh(\frac{1}{x}),\]
par exemple $min(1,x)$ ou bien $\frac{x}{1+x}$. La fonction $V : E\rightarrow \R_+$ est la fonction "coût" à minimiser. \\

La terminologie "recuit simulé" vient d'une technique métallurgique consistant à faire fondre de façon répétée le métal puis à le faire lentement refroidir pour en améliorer les propriétés. En effet, on va se donner un schéma de décroissance d'une quantité analogue à la température, que nous noterons $T_n$. Ce schéma est crucial pour que l'algorithme converge vers un minimum global de la fonction $V$ et ne reste pas piégé dans un de ses minima locaux. Vous pourrez choisir l'un des schémas suivants :
\begin{itemize}
\item décroissance logarithmique : \[T_n = \frac{C}{\log (n)}\]
\item recuit par palier : \[T_n = \frac{1}{k}\text{ pour } e^{(k-1)C}\leq n < e^{kC}\]
\end{itemize}

Voici l'algorithme :\\

\fbox{\begin{minipage}{0.9\textwidth}
Initialiser $X_0$. Choisir le nombre de pas de la marche aléatoire $m$. Pour $n$ allant de $1$ à $m-1$, répéter :
\begin{enumerate}
\item Choisir un voisin $y$ de $X_n$ aléatoirement.
\item Tirer $U\sim \mathcal{U}_{[0,1]}$.
\item Si $U<h(\exp(\frac{1}{T_n})(V(X_n)-V(y))\frac{N(y)}{N(X_n)})$, accepter $X_{n+1}=y$, sinon refuser i.e. $X_{n+1}=X_n$.
\end{enumerate}
\end{minipage}}\\
\\

Le problème auquel nous allons appliquer cet algorithme est celui d'un commerçant devant visiter un ensemble fini $E={X_1,...,X_N}$ de villes, une et une seule fois. Pour minimiser son temps et son argent, il souhait trouver le chemin $l$ qui minimise la distance, soit, avec nos notations :
\[V(l)=\sum_{j=1}^{N-1} d(X_{l(j)},X_{l(j+1)}),\]
où l'on voit un chemin comme une permutation de l'ensemble $E$. Nous travaillerons avec des villes disposées aléatoirement dans le carré $[0;1]\times [0;1]$.
 
\begin{enumerate}
\item Créer une fonction qui calcule le coût d'un chemin donné $l$.
\item Créer une fonction qui affiche un chemin donné $l$.
\item Choisir une loi de transition sur les chemins, et l'implémenter.
\item Implémenter l'algorithme du recuit simulé sur ce problème, à l'aide d'une fonction si possible. Afficher le chemin obtenu, ainsi que l'évolution de la longueur du chemin en fonction du nombre d'itérations . Qu'en pensez vous ? De combien d'itérations avez-vous besoin pour obtenir un chemin plausible ? 
\end{enumerate}



\part{Probabilités et processus aléatoires}

\chapter{Estimation}

\section{L'essentiel du cours}

On se fixe un espace probabilisé $(\Omega,\mathcal A ,\mathbb{P})$. Pour toute variable aléatoire $X$, on note $F_X(x)= \mathbb{P}(X\leq x)$ sa fonction de répartition, et $\phi_X(t)= \mathbb{E}(e^{itX})$ sa fonction caractéristique. On rappelle que $F_X$ est une fonction càd-làg, i.e. continue à droite et possédant une limite à gauche.

\subsection{Un rappel sur les convergences}

On rappelle quelques notions de convergence utilisées en probabilités.\\ 

\begin{definition}
Soit $(X_n)$ une suite de variables aléatoires.
\begin{itemize}
\item[$\bullet$] $X_n$ converge presque-sûrement vers $X$, ce que l'on note $X_n \overset{p.s.}{\longrightarrow}X$, si $\lim X_n(\omega) = X(w)$ pour tout $\omega\in \Omega \backslash \mathcal N$ où $\mathcal N$ est un ensemble négligeable, 
\item[$\bullet$] $X_n$ converge en probabilité vers $X$, ce que l'on note $X_n \overset{\mathbb P}{\longrightarrow} X$, si $\lim_{n \rightarrow \infty} \mathbb P (|X-X_n|>\varepsilon)= 0$ pour tout $\varepsilon>0$, 
\item[$\bullet$] $X_n$ converge en loi vers $X$, ce que l'on note $X_n \overset{\mathcal L}{\longrightarrow} X$, si $\lim F_{X_n}(x) = F_X(x)$ pour tout $x\in \R$,
\item[$\bullet$] Soit $p\geq 1$ et $X_n$ une suite de variables aléatoires dans $L^p(\Omega,\mathbb P)$. $X_n$ converge en norme $L^p$ vers $X$, ce que l'on note $X_n \overset{L^p}{\longrightarrow} X$, si $X \in L^p(\Omega,\mathbb P)$ et $\lim \mathbb{E}[|X_n-X|^p] = 0$.\\ 
\end{itemize}
\end{definition}

Le théorème de Lévy assure que $X_n \overset{\mathcal L}{\longrightarrow} X$ ssi on a convergence simple des fonctions caractéristiques sur $\R$, i.e. $\phi_{X_n} \overset{cvs}{\longrightarrow} \phi_X $.\\

Voici un récapitulatif des différentes convergences :
\[\begin{tikzcd}
     &              & L^q  &                   &          \\
     &              & \Downarrow &    \text{avec }q \geq p       &            \\
     &              & L^p     &               &            \\
     &              & \Downarrow &           &            \\
p.s. & \Rightarrow  & \mathbb P & \Rightarrow & \mathcal L
\end{tikzcd}\]

\subsection{Les théorèmes fondamentaux}

Les marins... Le cadre probabiliste formalise cette situation en postulant que les mesures sont des variables aléatoires $\theta_j$ identiquement distribuées (i.i.d.). On notera : 
\[\hat\theta_n = \frac{1}{n}\sum_{j=1}^n \theta_j.\]

\begin{thm}[LGN]
Si $\mathbb E |X|<\infty $ alors $\hat \theta_n \overset{p.s.}{\longrightarrow}\theta$.
\end{thm}

\begin{thm}[TCL]
Si $\mathbb E |X|<\infty $ et $\sigma^2=Var(X)<\infty$ alors \[\sqrt{n}\frac{(\hat \theta_n -\theta)}{\sigma}\overset{\mathcal L}{\longrightarrow}\mathcal{N}(0,1)\]
\end{thm}

\begin{lem}[Slutsky]
Si $X_n \overset{\mathcal L}{\longrightarrow} X$ et $Y \overset{p.s.}{\longrightarrow}a$ où $a$ est une constante, alors le couple $(X_n,Y_n) \overset{\mathcal L}{\longrightarrow} (X,a)$.
\end{lem}

Une application importante est que l'on peut se passer de connaître la variance théorique $\sigma^2$ et simplement l'estimer, car, dans les hypothèses du théorème Central-Limite, si on note $\hat\sigma_n^2 =\frac{1}{n}\sum (\theta_j -\hat{\theta}_n)^2$, le lemme de Slutsky assure que :
\[\sqrt{n}\frac{(\hat\theta_n - \theta)}{\hat\sigma_n} \overset{\mathcal L}{\longrightarrow} \mathcal N(0,1)\]

Ces théorèmes sont asymptotiques, et ne donnent pas d'informations sur le nombres d'observations nécessaires pour atteindre une précision donnée. Le théorème de Berry-Essen apporte une réponse à cette problématique.\\

\begin{thm}[Berry-Essen]
\end{thm}
 
Le théorème fondamental de la Statistique donne la convergence uniforme de la f.d.r. empirique vers la "vraie" f.d.r d'un échantillon. On se donne une suite $(X_j)_j$ de variables aléatoires i.i.d. et on note : 
\[\hat F_n(x) = \frac{1}{n} \sum_{j=1}^n 1_{X_j\leq x},\]
appelée fonction de répartition empirique.

\begin{thm}[TFS, Glivenko-Cantelli]
La fonction de répartition empirique $\hat F_n$ converge uniformément vers la fonction de répartition théorique $F$ avec probabilité $1$ :
\[\sup_{x\in \R} |F_n(x)-F(x)| \underset{n\to\infty}{\longrightarrow} 0.\]
\end{thm}

Pour la culture, le théorème de Donsker affirme que le processus $W_n(t) = \sqrt{n}(\hat F_n(t)-F(t))$ converge en loi vers un pont brownien, dans l'espace des fonctions càd-làg munie de la topologie de Storokhod. On a de plus une inégalité plus précise, qui permet d'estimer la vitesse de convergence de la f.d.r. empirique, appelée inégalité DKW (Dvoretzky-Kiefer-Wolfowitz) :
\[\forall \varepsilon>0, \mathbb{P}(||\hat F_n -F||_\infty>\varepsilon)\leq e^{-2n\varepsilon^2}.\] 
Le livre de Van Der Vaart et Wellner \cite{VDVWellner} et le livre de Van Der Vaart \cite{VDV} sont de bonnes références pour ces résultats. 

\section{Quelques exemples}

\subsection{Théorème de Moivre-Laplace}

Soient $X_i$ des variables aléatoires i.i.d. suivant une loi de Bernoulli de paramètre $p\in (0,1)$ et 
\[S_n=\frac{1}{n}\sum_{i=1}^n X_i.\]

\begin{enumerate}
\item Quelle loi suit $S_n$ ?
\item Calculer l'espérance et la variance de $S_n$.
\item Vers quoi tend \[\sqrt{n}\frac{S_n - p}{p(1-p)}\quad \text{?}\]
Rappeler le mode de convergence et sa définition, ainsi que les différents modes de convergence que vous connaissez, et leurs liens logiques.
\end{enumerate}

\subsection{Moindres carrés}

On se place dans l'espace euclidien $(\mathbb R^n, \langle \ , \ \rangle) $, on notera $||x||_2 = \sqrt{\langle x,x\rangle}$, et on fixe $x_1,...,x_N \in \R^n$. On définit 
\[L\left\{ \begin{array}{rcl} \R^n & \rightarrow & \R \\ x &  \mapsto & \sum_{i=1}^N ||x-x_i||_2^2 \end{array}\right.\]

Montrer que $L$ admet un unique minimum sur $\R^n$, et le calculer explicitement en fonction des $x_i$.

\subsection{Moyenne et variances empiriques}
Soient $X_i$ des variables aléatoires i.i.d. d'espérance $\mu\in\R$ et de variance $\sigma^2>0$. On définit 
\[\begin{array}{rcl} \overline X_n = \frac{1}{n}\sum_{i=1}^n X_i, & S^2_n = \frac{1}{n-1}\sum_{i=1}^n (X_i -\overline{X}_n)^2, & \hat{\sigma}^2_n = \frac{1}{n}\sum_{i=1}^n (X_i -\overline{X}_n)^2  \end{array}.\]

\begin{enumerate}
\item Calculer l'espérance et la variance de chacune de ces statistiques.
\item De quoi ces statistiques sont-elles les estimateurs ? Préciser leurs propriétés.
\end{enumerate}


\section{Généralités sur l'estimateur du maximum de vraisemblance}
\begin{enumerate}
\item Rappeler les propriétés de l'EMV.
\item Soient $X_j$ des variables exponentielles indépendantes de paramètre $\theta>0$, non-observées, et $T$ un instant de censure. Soit $\mathcal E^n$ l'expérience engendrée par l'observation du $n$-échantillon $X_j^*=\min{\{T,X_j\}}$. Donner une mesure qui domine le modèle et calculer sa vraisemblance.
\item Montrer que l'estimateur du maximum de vraisemblance ne dépend pas du choix de la mesure dominante.  
\end{enumerate}
\section{Exemples de calculs de maximum de vraisemblance}

Pour chaque loi, on considère un $n$-échantillon tiré de façon i.i.d selon cette loi. Proposer un espace des paramètres donnant un modèle identifiable. Donner une mesure dominante si possible. Calculer la vraisemblance du modèle, ainsi que la log-vraisemblance, donner les équations de vraisemblance et déterminer , s'il existe, un estimateur du maximum de vraisemblance.\\

\begin{enumerate}
\item Modèle gaussien standard, de densité par rapport à la mesure de Lebesgue \[f_\theta(x)=\frac{1}{\sqrt{2\pi \sigma^2}} \exp{-\frac{1}{2\sigma^2}}(x-\mu)^2\quad, \theta=(\mu,\sigma).\]
\item Modèle de Bernoulli \[\mathbb P_\theta(X=1)=1-\mathbb P(X=0)=\theta.\]
\item Modèle de Laplace, où $\sigma>0$ est connu, de densité par rapport à la mesure de Lebesgue \[f_\theta(x)=\frac{1}{2\sigma}\exp{(-\frac{|x-\theta|}{\sigma})}.\]
\item Modèle uniforme, de densité par rapport à la mesure de Lebesgue \[f_\theta(x)=\frac{1}{\theta}1_{[0,\theta]}(x).\]
\item Modèle de Cauchy, de densité par rapport à la mesure de Lebesgue \[f_\theta(x)=\frac{1}{\pi(1-(x-\theta)^2)}.\]
\item Modèle de translation. On considère la densité \[h(x)=\frac{ e^{-\frac{|x|}{2}}}{2\sqrt{2\pi|x|}}.\] Le modèle de translation par rapport à la densité $h$ est le modèle dominé par la mesure de Lebesgue sur $\R$ de densités 
\[f_\theta(x)=h(x-\theta)\quad, x\in R,\theta \in \R.\]
\end{enumerate}

\section{Méthode des moments}
\begin{enumerate}
\item Calculer des estimateurs des moments d'ordre $1$ et $2$ pour l'expérience engendrée par l'observation d'un $n$-échantillon de variables exponentielles de paramètre $\theta>0$. Donner l'asymptotique des ces deux estimateurs.
\item On considère le modèle de translation associé à la famille des lois de Cauchy :
\[f_\theta(x)=\frac{1}{\pi(1+(x-\theta)^2)}\quad, x\in\R.\]
On note $g$ la fonction signe, qui vaut $1$ si $x>0$, $-1$. Trouver un estimateur pour $\theta\mapsto \mathbb E[g(X_1)]$ et donner ses propriétés.
\end{enumerate}

\section{Estimation de la fonction de répartition}

On se donne un $n$-échantillon $X_1$,..., $X_n$ i.i.d suivant une loi donnée par la même fonction de répartition (f.d.r) $F$ sur $\R$. $\mathcal F$ dénote l'ensemble des fonctions de répartition sur $\R$.

\begin{enumerate}
\item Décrire l'expérience statistique.
\item Le modèle est-il dominé ?
\item On veut estimer $F(x)=\mathbb P(X\leq x)$.
	\begin{enumerate}
	\item On pose $\hat F_n(x)=\frac{1}{n}\sum_{i=1}^n 1_{X_i\leq x}$. Calculer $\mathbb E[\hat F_n(x)]$ et $V[\hat F_n(x)]$.
	\item Montrer que $\hat F_n(x)$ converge presque-sûrement vers $F(x)$.
	\item Montrer que, si $l(x,y)=(x-y)^2$ est la perte quadratique, \[\sup_{F\in \mathcal F} \mathbb E[l(\hat F_n(x),F(x))]=\frac{1}{4n}.\]
	\item En déduire que $\hat F_n(x)$ converge uniformément en norme $\mathcal L^2$ vers $F(x)$, et donc en probabilité.
	\end{enumerate}
\item \begin{enumerate}
	\item Montrer que \[\mathbb P(|\hat F_n(x) - F(x)|>t)\leq \frac{1}{t^2}Var[\hat F_n (x)]\leq \frac{1}{4nt^2}\]
	\item Soit $\alpha\in ]0;1[$. Déterminer 
	\[t_{\alpha,n}=\inf \{t>0 \ : \ \frac{1}{4nt^2}\leq \alpha\}\]
	et en déduire un intervalle de confiance pour $F(x)$ de niveau $1-\alpha$.
	\item Comment interpréter $I_{n,\alpha}$ ? Quelle est sa précision ?
	\end{enumerate}
\item On pose $\xi_n = \sqrt{n}\frac{\hat F_n(x) - F(x)}{\sqrt{\hat F_n(x)(1-\hat F_n(x))}}$.
	\begin{enumerate}
	\item Déterminer la limite en loi de $\xi_n$.
	\item On note $J_{n,\alpha}$ l'intervalle $[-\phi^{-1}(1-\frac{\alpha}{2});\phi^{-1}(1-\frac{\alpha}{2})]$.Calculer la limite de $\mathbb P(\xi_n\in J_{n,\alpha})$ lorque $n$ tend vers $\infty$.
	\item Donner un intervalle de confiance asymptotique pour $J_{n,\alpha}$, ainsi que sa précision asymptotique.
	\end{enumerate}
\item Soient $Y_j$ des variables aléatoires réelles indépendantes centrées : $\mathbb E Y_j = 0 $ et bornées : $a_j \leq Y_j \leq b_j$. On veut démontrer ce que l'on appelle l'\textit{inégalité de Hoeffding} : pour tout $t>0$, 
\[\mathbb P(\sum Y_j <t )\leq e^{-st}\prod e^{\frac{s^2(b_j-a_j)^2}{8}}\quad\forall s >0.\]
On pose $\Phi_Y(s)=\log \mathbb E[e^{s(Y-\mathbb E Y)}]$. 
	\begin{enumerate}
	\item Montrer que \[\Phi''_Y(s)=e^{-\Phi_Y(s)}\mathbb E[Y^2 e^{sY}]-e^{-2\Phi_Y(s)}\mathbb (\mathbb E[Ye^{sY}])^2.\]
	\item On définit une nouvelle mesure de probabilité par $\mathbb Q(A)= e^{-\Phi_Y(s)}\mathbb E[e^{sY}1_A]$ pour tout borélien $A$. Comment interpréter $ \Phi''_Y(s)$ dans ce cadre ?
	\item Montrer alors que $\Phi_Y(s)\leq s^2 \frac{(b-a)^2}{8}$.
	\item En déduire l'inégalité de Hoeffding.
	\end{enumerate}

\item \begin{enumerate}
	\item Soient $X_j$ des v.a. de Bernoulli de paramètre $p$ et $\overline X_n = \frac{1}{n}\sum_{j=1}^n X_j$, montrer que 
		\[\mathbb P(|\overline X_n-p|>t)\leq 2e^{-2nt^2}.\] 
	\item En déduire un intervalle de confiance de niveau $1-\alpha$ pour $F(x)$.
	\end{enumerate}
\item Comparer les différents intervalles de confiance que vous avez obtenu.
\end{enumerate}


\section{Intervalles de confiance}
\begin{enumerate}
\item 
Soient $\mu\in \R$, $\sigma>0$ et $\alpha\in (0,1)$.
On observe un $n$-échantillon $\underline x=(X_1,...,X_n)$ de variables iid de loi normale $\mathcal N(\mu ,\sigma^2)$.
\begin{enumerate}
\item Donner un intervalle de confiance de niveau $\alpha$ pour $\mu$, si $\sigma^2$ est connu, puis si $\sigma^2$ est inconnu.
\item Donner un intervalle de confiance de niveau $\alpha$ pour $\sigma^2$, si $\mu$ est connu, puis si $\mu$ est inconnu.
\end{enumerate}

\item On observe un $n$-échantillon $\underline Y=(Y_1,...,Y_n)$ de variables iid suivant une loi de Bernoulli de paramètre $0<p<1$ inconnu. Donner un intervalle de confiance pour $p$ au niveau $\alpha$.

\end{enumerate}

\section{Estimateur du maximum de vraisemblance}
On observe un $n$-échantillon $\underline x=(X_1,...,X_n)$ de variables iid suivant une loi $\mathbb{P}_\theta$ de densité :
\[f_\theta (x ) = \frac{2}{\sqrt{\pi}\theta^{3/2}}x^2 e^{-\frac{x^2}{\theta}}, \text{ pour }\theta >0. \]

\begin{enumerate}
\item  Décrire le modèle statistique engendré par $\underline x$.
\item Calculer un estimateur du maximum de vraisemblance, que l'on notera $\hat \theta$.
\item Examiner les qualités suivantes de $\hat \theta$ : efficacité, biais et convergence.
\end{enumerate}

On observe un $n$-échantillon $\underline x=(X_1,...,X_n)$ de variables iid suivant une loi exponentielle de paramètre $\lambda >0$, de densité
\[g_\lambda (x)=\lambda e^{-\lambda x}1_{x\geq 0}.\]
\begin{enumerate}
 \item Décrire le modèle statistique engendré par $\underline x$.
\item Calculer la vraismeblance du modèle.
\item Donner un estimateur du maximum de vraisemblance pour $\lambda$.
\end{enumerate}



\section{Estimation de la fonction de répartition}

On se donne un $n$-échantillon $X_1$,..., $X_n$ i.i.d suivant une loi donnée par la même fonction de répartition (f.d.r) $F$ sur $\R$. $\mathcal F$ dénote l'ensemble des fonctions de répartition sur $\R$.

\begin{enumerate}
\item Décrire l'expérience statistique.
\item Le modèle est-il dominé ?
\item On veut estimer $F(x)=\mathbb P(X\leq x)$.
	\begin{enumerate}
	\item On pose $\hat F_n(x)=\frac{1}{n}\sum_{i=1}^n 1_{X_i\leq x}$. Calculer $\mathbb E[\hat F_n(x)]$ et $V[\hat F_n(x)]$.
	\item Montrer que $\hat F_n(x)$ converge presque-sûrement vers $F(x)$.
	\item Montrer que, si $l(x,y)=(x-y)^2$ est la perte quadratique, \[\sup_{F\in \mathcal F} \mathbb E[l(\hat F_n(x),F(x))]=\frac{1}{4n}.\]
	\item En déduire que $\hat F_n(x)$ converge uniformément en norme $\mathcal L^2$ vers $F(x)$, et donc en probabilité.
	\end{enumerate}
\item \begin{enumerate}
	\item Montrer que \[\mathbb P(|\hat F_n(x) - F(x)|>t)\leq \frac{1}{t^2}Var[\hat F_n (x)]\leq \frac{1}{4nt^2}\]
	\item Soit $\alpha\in ]0;1[$. Déterminer 
	\[t_{\alpha,n}=\inf \{t>0 \ : \ \frac{1}{4nt^2}\leq \alpha\}\]
	et en déduire un intervalle de confiance pour $F(x)$ de niveau $1-\alpha$.
	\item Comment interpréter $I_{n,\alpha}$ ? Quelle est sa précision ?
	\end{enumerate}
\item On pose $\xi_n = \sqrt{n}\frac{\hat F_n(x) - F(x)}{\sqrt{\hat F_n(x)(1-\hat F_n(x))}}$.
	\begin{enumerate}
	\item Déterminer la limite en loi de $\xi_n$.
	\item On note $J_{n,\alpha}$ l'intervalle $[-\phi^{-1}(1-\frac{\alpha}{2});\phi^{-1}(1-\frac{\alpha}{2})]$.Calculer la limite de $\mathbb P(\xi_n\in J_{n,\alpha})$ lorque $n$ tend vers $\infty$.
	\item Donner un intervalle de confiance asymptotique pour $J_{n,\alpha}$, ainsi que sa précision asymptotique.
	\end{enumerate}
\item Soient $Y_j$ des variables aléatoires réelles indépendantes centrées : $\mathbb E Y_j = 0 $ et bornées : $a_j \leq Y_j \leq b_j$. On veut démontrer ce que l'on appelle l'\textit{inégalité de Hoeffding} : pour tout $t>0$, 
\[\mathbb P(\sum Y_j <t )\leq e^{-st}\prod e^{\frac{s^2(b_j-a_j)^2}{8}}\quad\forall s >0.\]
On pose $\Phi_Y(s)=\log \mathbb E[e^{s(Y-\mathbb E Y)}]$. 
	\begin{enumerate}
	\item Montrer que \[\Phi''_Y(s)=e^{-\Phi_Y(s)}\mathbb E[Y^2 e^{sY}]-e^{-2\Phi_Y(s)}\mathbb (\mathbb E[Ye^{sY}])^2.\]
	\item On définit une nouvelle mesure de probabilité par $\mathbb Q(A)= e^{-\Phi_Y(s)}\mathbb E[e^{sY}1_A]$ pour tout borélien $A$. Comment interpréter $ \Phi''_Y(s)$ dans ce cadre ?
	\item Montrer alors que $\Phi_Y(s)\leq s^2 \frac{(b-a)^2}{8}$.
	\item En déduire l'inégalité de Hoeffding.
	\end{enumerate}

\item \begin{enumerate}
	\item Soient $X_j$ des v.a. de Bernoulli de paramètre $p$ et $\overline X_n = \frac{1}{n}\sum_{j=1}^n X_j$, montrer que 
		\[\mathbb P(|\overline X_n-p|>t)\leq 2e^{-2nt^2}.\] 
	\item En déduire un intervalle de confiance de niveau $1-\alpha$ pour $F(x)$.
	\end{enumerate}
\item Comparer les différents intervalles de confiance que vous avez obtenu.
\end{enumerate}


\section{Théorème de Cochran et applications}

Voici un énoncé simplifié du théorème de Cochran.
\begin{thm}[Cochran]
Soit $E=(\R^n,\langle \ ,\ \rangle)$ l'espace euclidien usuel, et $F$ un sous-espace vectoriel de $E$ de dimension $p\leq n$. Notons $P$ la projection orthogonale sur le sous-espace $F$. Soit $\underline x$ un vecteur gaussien de $E$ centré réduit. \\
Les vecteurs $P\underline x$ et $P^{\perp}\underline x$ sont indépendants, gaussiens, centrés et de matrice de variance-covariance respectives $P$ et $P^{\perp}$.\\
Les variables aléatoires $||P\underline x||^2$ et $||P^{\perp}\underline x||^2$ sont indépendantes et suivent une loi du $\chi^2$ à $p$ et $n-p$ degrés de liberté, respectivement.
\end{thm}

\begin{enumerate}
\item Démontrer le théorème.
\item Soient $X_j$ un $n$-échantillon gaussien i.i.d d'espérance $\mu$ et de variance $\sigma^2$. On note
\[\overline X_n = \frac{1}{n}\sum_{j=1}^n X_j\text{  et  } s^2_n=\frac{1}{n-1} \sum_{j=1}^n (X_j-\overline X_n)^2.\]
Démontrer que $\overline X_n $ suit un loi normale $\mathcal N(\mu, \frac{\sigma^2}{n})$ et que $(n-1)\frac{s^2_n}{\sigma^2}$ suit une loi du $\chi^2$ à $n-1$ degrés de liberté. En déduire la loi de $\sqrt{n}\frac{\overline X_n -\mu}{s^2_n}$.
\end{enumerate}

\section{Un modèle non-linéaire}

Soit $f : \R^n \times \R^k \rightarrow \R^n$ une fonction de classe $\mathcal C^2$ que l'on suppose connue. Soit le modèle 
\[y=f(X,\alpha)+\epsilon \text{  et  } \epsilon \sim \mathcal N(0,\sigma^2 I_n).\] 
On cherche à estimer \[\theta=(\alpha, \sigma^2) \in \Theta\subset \R^{k+1}.\]
On note $L(\alpha)=||y-f(X,\alpha)||^2$.\\
\begin{enumerate}
\item Définir le modèle, et calculer la vraisemblance.
\item Montrer que maximiser la vraisemblance est équivalent à minimiser $L(\alpha)$. 
\item Calculer l'information de Fisher du modèle.
\end{enumerate}

\section{Maximum de vraisemblance et séries temporelles}

Soient $\lambda\in \R$ tel que $|\lambda|<1$, $c\in \R$ et $\sigma^2 >0$. On observe un échantillon $\{Y_t\}_{t\leq T}$ que l'on pense suivre le modèle $AR(1)$
\[Y_t = c +\lambda Y_{t-1} +\epsilon_t  \text{  où les } \epsilon_t \sim \mathcal N(0,\sigma^2)\]
sont des variables i.i.d.\\
On cherche à estimer 
\[\theta = (c,\lambda,\sigma^2)^{T}\in\Theta\subset \R^3.\]
\begin{enumerate}
\item Calculer $\mathcal L(Y_1;\theta)$, $\mathcal L(Y_t|Y_{t-1};\theta)$, et en déduire $\mathcal L(Y_2,Y_1;\theta)$.
\item Calculer la vraisemblance du modèle $\mathcal L(Y_1,...,Y_n|\theta)$. 
\item Calculer la matrice de variance-covariance du processus $AR(1)$ gaussien. On la note $\Omega$.
\item Réécrire la log-vraisemblance du modèle en utilisant $\Omega$. Quel est le lien avec la question $2$ ?
\item Déterminer un estimateur du maximum de vraisemblance.
\item Refaire l'exercice pour le modèle $MA(1)$ gaussien
\[Y_t = c + \epsilon_t -\theta \epsilon_{t-1}\text{  où } \epsilon_t \sim \mathcal N(0,\sigma^2) \text{ i.i.d. }\]
\item En cas de forme de motivation extrême, le faire pour le modèle $ARMA(p,q)$ gaussien
\[Y_t=c+\sum_{j=1}^p \lambda_{j}Y_{t-j}+\epsilon_t +\sum_{j=1}^q \theta_{j}\epsilon_{t-j}\] 
avec $\epsilon_t \sim \mathcal N(0,\sigma^2)$ i.i.d.
\end{enumerate}

\section{Test du $\chi^2$}
\begin{enumerate}
\item Soit $(X_k , Y_k )_{k=1,...,n}$ un $n$-échantillon d’une loi $Q = (q_{ij} )_{(i,j)\in \{1,...,I\}^2}$ sur $\{1, . . . , I\}^2$ ,
dont les marginales sont égales.
Soit, pour tout $k = 1, . . . , n$, le vecteur aléatoire
\[Z_k = (1_{X_k =i} - 1_{Y_k =i} )_{1\leq i \leq I} .\]
\begin{enumerate}
 \item Quelle est la matrice de covariance $\Gamma$ de $Z_k$ ?
\item On suppose $\Gamma$ inversible et on note son inverse
\[\Gamma^{-1} = (\Gamma^{ij})_{1\leq i,j\leq I-1} .\]

Soient, pour tout $i, j = 1, . . . , I,$
\[\begin{array}{l} N_{ij} =  \sum_{k = 1}^{n} 1_{X_k =i,Y_k =j} \\

	N_{i.} = \sum_{k = 1}^{n} 1_{X_k =i} \\ 

	N_{.j} = \sum_{k = 1}^{n} 1_{Y_k =j} \\
\end{array}
\]
Quelle est la loi asymptotique de

\[\frac{1}{n}\sum_{1\leq i,j \leq I}(N_{i.} - N_{.i} )(N_{j.} - N_{.j} )\Gamma^{ij} \text{   ?}\]
\end{enumerate}

\item On ne suppose plus a priori que les marginales soient égales. On observe $(X_k , Y_k )$
décrit comme ci-dessus. Soit $V$ la matrice $(V_{ij} )_{1\leq i,j\leq I-1}$ définie par
\[ nV_{ii} = N_{i. }+ N_{.i} - 2N_{ii} , \]
et pour tout $i \neq  j$,
\[nV_{ij} = -(N_{ij} + N_{ji} ).\]
\begin{enumerate}
\item Montrer que, sous l’hypothèse d’égalité des marginales, $V$ converge vers $\Gamma$.
\item Soit $(V^{ij} )_{ 1\leq i,j\leq I-1} $ l’inverse de V . Montrer que
\[\Delta= \frac{1}{n}\sum_{i,j} (N_{i.} - N_{.i} )(N_{j.} - N_{.j} )V _{ij}\]
converge vers une loi $\chi^2 (I - 1)$.
\end{enumerate}

\item Quel test peut-on construire ?
\item Appliquer ce test aux données suivantes. On évalue le degré de vision des deux yeux de 7477 femmes agées de 30 à 40 ans en le classifiant en $4$ groupes ($1$ à $4$, du meilleur au pire). On obtient \\

\begin{center}
\begin{tabular}{|c||c|c|c|c|}
\hline
oeildroit | oeilgauche &	1 & 	2	&	3 	& 	4 	\\
\hline
1			&  1520 	  &	266	& 124 		&	66	\\
\hline
2			&  234 	  & 	1512 	& 432		& 	78	\\
\hline
3			& 117 	&  362		& 	1772	& 	205	\\
\hline
4			& 36 	& 82 & 179 & 492 \\
\hline
\end{tabular}
\end{center}
\end{enumerate}

%\section{Modèles à énergie}

Soit $(\Omega,\mathcal{A},\lambda)$ un espace mesuré, représentant l'espace des configurations d'un système. On suppose de plus que l'on dispose d'une fonction 
\[ H : \Omega \rightarrow \R \]
que l'on appellera énergie.\\

Soit $\beta>0$. La loi de probabilité qui gouverne le système à la température $\frac{1}{\beta}$ est donnée par :
\[P_\beta (X\in A) = \int_A \frac{e^{-\beta H(\sigma)}{Z_\beta} d\lambda(\sigma),\]
où $Z_\beta = \int_\Omega e^{-\beta H(\sigma)} d\lambda(\sigma)$ est la constante de normalisation du système.

\begin{enumerate}
\item Décrire le modèle statistique.
\item On se place dans le cas $(\R^q\times\R^p, \mathcal{B},Leb_{p+q})$, et $H(q,p) = \frac{1}{2} p^2 + V(q) $.
\item Matrices aléatoires.
\item Réseau de neurones.
\end{enumerate}


\chapter{Tests}
\section{Principe de Neyman : décision à $2$ points}

\begin{enumerate}

\item Soit $f$ la densité d'une loi de probabilité sur $\R$, et $\mathcal E$ l'expérience statistique engendré par un $n$-échantillon de loi $p_\theta(x)=f(x-\theta)$. On suppose que $\Theta =\{0, \theta_0\}$ avec $\theta_0\neq 0$. On veut tester $H_0| \theta = 0$ contre $H_1| \theta= \theta_0$.
\begin{enumerate}
\item Décrire l'expérience statistique et donner la vraisemblance du modèle.
\item Donner la zone de rejet du test de Neyman-Pearson de niveau $\alpha$ associé à $H_0$ et $H_1$.
\end{enumerate}

\item L'expérimentateur observe une seule réalisation d'une v.a. $X$ de loi de Poisson de paramètre $\theta>0$. On veut tester $H_0| \theta = \theta_0$ contre $H_1| \theta= \theta_1$, où $\theta_0\neq \theta_1$.
\begin{enumerate}
\item Donner la zone de rejet du test de Neyman-Pearson de niveau $\alpha$ associé.
\item Sachant que $\mathbb P_{\theta_0}(X>9)=0.032$ et $\mathbb P_{\theta_1}(X>8)=0.068$, donner une zone de rejet explicite pour $\alpha = 0.05 = 5\%$. Le test est-il optimal ?
\end{enumerate}

\end{enumerate}

\section{Neyman-Pearson : familles à rapport de vraisemblance monotone}

\begin{enumerate}
\item Soit $\mathcal E$ l'expérience statistique engendrée par un $n$-échantillon de loi normale $\mathcal N (\theta,\sigma^2)$, où $\sigma^2$ est connu, et $\theta\in \Theta =\R$. On souhaite tester $H_0| \theta = \theta_0$ contre $H_1| \theta= \theta_1$, où $\theta_0<\theta_1$.
\begin{enumerate}
\item Décrire le modèle ainsi que la vraisemblance. On choisira la mesure de Lebesgue comme mesure dominante.
\item Calculer le rapport de vraisemblance \[\frac{f(\theta_1,Z)}{f(\theta_0,Z)}.\]
\item Donner la zone de rejet pour le test de Neyman-Pearson associé.
\end{enumerate}

\item Pour la même expérience statistique, on a un test optimal (uniformément plus puissant ) de $H_0$ contre $H_1$ donné par la région de rejet 
\[\mathcal R = \{\overline X_n >c\}\]
où $c$ est solution de $\mathbb P_{\theta_0}(\overline X_n > c)=\alpha$.
\begin{enumerate}
\item Calculer explicitement la valeur de la constante $c=c(\theta_0,\alpha)$.
\item Calculer la puissance de ce test.
\end{enumerate}
\end{enumerate}

\section{Exercice}
L'expérimentateur observe $2$ échantillons indépendants $X_1,...,X_n$ et $Y_1,...,Y_m$ de tailles distinctes $n\neq m$, de lois respectives $\mathcal N(\mu_1,\sigma^2_1)$ et $\mathcal N(\mu_2,\sigma^2_2)$. Il souhaite tester
\[H_0\ : \ \mu_1=\mu_2 \quad \text{contre} \quad H_1  : \ \mu_1\neq \mu_2 .\]
Si $s_{n,1}^2 = \frac{1}{n}\sum_{j=1}^n (X_j-\overline X_n)^2$ et $s_{m,2}^2 = \frac{1}{m}\sum_{j=1}^m (Y_j-\overline Y_m)^2$, construire un test basé sur la statistique 
\[T_{n,m}=\frac{\overline X_n - \overline Y_m}{\sqrt{s_{n,1}^2+s_{m,2}^2}}\]
et étudier sa consistance.


\section{Neyman-Pearson : loi exponentielle}

On observe un $n$-échantillon $\underline x=(X_1,...,X_n)$ de variables iid de loi exponentielle de paramètre $\lambda>0$, de densité 
\[x\mapsto \lambda \exp(-\lambda x)1_{x\leq 0}.\]
\begin{enumerate}
\item Rappeler l'espérance et la variance (les calculer si besoin) d'une loi exponentielle de paramètre $\lambda$. On rappelle que $2\lambda \sum_{j=1}^n X_j$ suit alors une loi du $\chi^2$ à $2n$ degrés de liberté.
\item Ecrire le modèle statistique engendré par l'observation $ \underline x$.
\item Calculer l'estimateur du maximum de vraisemblance $\hat \lambda^{MV}$ de $\lambda$.
\item Montrer que $\hat \lambda^{MV}$ est asymptotiquement normal, et calculer sa variance limite.
\item Soient $0<\lambda_0 < \lambda_1$. Construire un test d'hypothèse de 
\[H_0 : \lambda =\lambda_0 \text{ contre } H_1 : \lambda = \lambda_1\]
de niveau $\alpha$ et uniformément plus puissant. Expliciter le choix du seuil définissant la région critique. Montrer que le test est consistant, i.e. que l'erreur de seconde espèce du test tend vers $0$ lorsque $n\rightarrow \infty$.
\end{enumerate}

\section{ Tests du $\chi^2$}

On considère une variable qualitative $X$, à valeur dans un ensemble fini $E=\{1,..,d\}$. Les lois de probabilité de telles v.a. sont entièrement décrites par le vecteur de probabilité $(p_1,...,p_d)^T$, où $p_j=\mathbb P(X=j)$. On confondera donc les lois de probabilités de $E$ avec 
\[\mathfrak M_d=\{p=(p_1,...,p_d)^T : 0\geq p_j\geq 1 \text{ et } \sum p_j = 1\}.\]

\begin{enumerate}
\item \textbf{Test d'adéquation du $\chi^2$.} On observe un $n$-échantillon de loi $p$ et l'on souhaite tester $p=q$ contre $p\neq q$, où $q\in \mathfrak M_d$ est une loi fixée.
\begin{enumerate}
\item Décrire le modèle statistique.
\item On définit les fréquences empiriques
\[\ \hat p_{n,l} = \frac{1}{n}\sum_{j=1}^n 1_{X_j=l}\quad\text{pour }l=1,...,d.\]
Donner la limite du vecteur $\hat p_n=(\hat p_{n,l} )_{l=1,d}^T$ pour la topologie de la convergence en probabilité sous $\mathbb P_p$.
\item On définit
\[U_n(p)=\sqrt{n}(\frac{\hat p_{n,l}-p_l}{\sqrt{p_l}})_{l=1,d}^T.\]
Donner la limite en loi de chaque composante de $U_n(p)$ sous $\mathbb P_p$. Que peut-on dire a priori de la limite en loi de $U_n(p)$ ? Pourquoi ?
\item On définit
\[Y_l^j = \frac{1}{\sqrt{p_l}}(1_{X_j=l}-p_l).\]
Si $Y_j$ désigne le vecteur $(Y_1^j, ... ,Y_d^j) $, montrer que $\frac{1}{\sqrt{n}}\sum Y_j = U_n(p)$.
\item Calculer $E[Y^j _l ]$, et $E[Y^j _l Y_{l'}^j]$. Que valent les composantes de la matrice $V(p)= I_d- \sqrt{p}\sqrt{p}^T$, où $\sqrt{p}=(\sqrt{p_1}, ... , \sqrt{p_d})^T$ ?
\item En déduire la limite en loi sous $\mathbb P_p$ de $U_n(p)$ et de $||U_n(p)||^2$, le carré de sa norme euclidienne.
\item Soient $p,q\in \mathfrak M_d$ tels que les coefficients de $q$ soient tous non nuls. On définit :
\[\chi^2 (p,q) = \sum_{l=1}^d\frac{(p_l-q_l)^2}{q_l}.\]
Cette quantité est appelée ''distance du $\chi^2$'' bien que ce ne soit pas une distance ! Toutefois, $\chi^2(p,q)=0$ ssi $p=q$.\\
Montrer que $n\chi^2(\hat p_n,p)= ||U_n(p)||^2$.
\item On définit, pour $\alpha \in (0,1)$, la zone de rejet
\[\mathcal R_{n,\alpha}=\{n\chi^2(\hat p_n,p)\geq q_{1-\alpha, d-1}^{\chi^2}\},\]
où $q_{1-\alpha, d-1}^{\chi^2}$ est le quantile d'ordre $1-\alpha$ de la loi du $\chi^2$ à $d-1$ degrés de liberté.\\
Montrer que le test associé est asymptotiquement de niveau $\alpha$ et est asymptotiquement consistant.
\item\text{Application numérique.} On décrit ici l'expérience de Mendel. Le croisement des pois fait apparaître $4$ phénotypes, distibués selon une loi multinomiale de paramètre
\[ q =(\frac{9}{16},\frac{3}{16}),\frac{3}{16},\frac{1}{16}).\]
Pour $n=556$ observations, Mendel rapporte les observations suivantes : les phénotypes se répartissent selon $(315,101,108,32)$. Sachant que le quantile d'ordre $0.95$ de la loi du $\chi^2$ à $3$ degrés de liberté vaut $0.7815$, accepter vous le test $p=q$ contre $p\neq q$.
\end{enumerate}

\end{enumerate}


\bibliographystyle{plain}
\bibliography{biblio}
\nocite{*}
\end{document}




























