\section{Exercice 1}
\begin{enumerate}
\item Que font les commandes $pnorm$, $qnorm$ , $rnorm$, $dnorm$ ? Utilisez l'aide de \textit{R}.
\item Tracer la fonction de densité de la loi normale. Faites varier les paramètres et afficher les différentes courbes sur le même graphique. 
\item Simuler 2 vecteurs $X$ et $Y$ contenant chacun $N=100$ variables indépendantes identiquement distribuées suivant une loi normale $\mathcal N (0,1)$.
\item Afficher les points de coordonnées $(X[j],Y[j])$ dans le plan, pour $j$ allant de $1$ à $100$.
\item Tracer la fonction de répartition empirique des $X[j]$.
\item Soit $\mathcal E$ une v.a. de loi exponentielle de paramètre $1$, et $U$ une v.a. suivant une loi uniforme sur $[0,2\pi]$. On pose 
\[(X,Y)=(\sqrt{\mathcal E}\cos (U), \sqrt{\mathcal E} \sin (U)).\]
Quelle est la loi du couple $(X,Y)$ ? (Vous pouvez le prouvez, ou observer grâce à \textit{R} ce qu'il se passe en simulant ces variables et en les traçant.)
\end{enumerate}

\section{Exercice 2}

\begin{enumerate}
\item Une table est déjà en mémoire dans \textit{R} : la table \textit{stackloss}. Analyser la rapidement.
\item Tracer \textit{stack.loss} en fonction de \textit{Air.Flow}. Qu'en pensez-vous ?
\item Effectuer la régression linéaire de \textit{stack.loss} en fonction des autres variables. Quelles sont celles qui sont significatives ?
\end{enumerate}

\section{Exercice 3}

Le fichier \textit{ozone.dta} contient les variables suivantes, pour une série de journées (qui sont ici nos individus) :
\begin{itemize}
\item l’identifiant de la journée,
\item le maximum d’ozone (variable maxO3)
\item l’heure à laquelle le maximum d’ozone a été obtenu (heure),
\item les températures à 6h, 9h, 12h, 15h, 18h (resp. T6 à T18)
\item la nébulosité à 6h, 9h, 12h, 15h, 18h (resp. Ne6 à Ne18)
\item la projection du vent sur l’axe est-ouest à 12h (Vx),
\item le maximum d’ozone de la veille (maxO3v).
\end{itemize}
Le but est de modéliser la valeur des pics d’ozone en fonction de grandeurs physiques facilement mesurables (température, heure, nébulosité, vent) afin d’avoir des approximations de la qualité de l’air faciles
et rapides à obtenir.\\


\begin{enumerate}
\item Importer la table, et afficher un résumé de ce qu'elle contient.
\item Tracer \textit{maxO3} en fonction de \textit{T12}, puis effectuer un régression linéaire. Ajouter la droite de régression sur le graphique. Soignez la présentation.
\item Afficher les résultats de la régression.
\item Extraire les résidus et tracer leur densité estimée.
\item Effectuer la régression de \textit{maxO3} sur toutes les variables, et supprimer récursivement celles qui ne sont pas significatives, jusqu'à ce qu'elles le soient toutes.
\end{enumerate}

\section{Exercice 4}
\begin{enumerate}
\item Sur le site \textit{data.gouv.fr}, vous pourrez trouver des tables de données publiques en libre accès. Choisissez un thème qui vous intéresse, puis une table en conséquence. Télécharger là.
\item Les tables sont souvent au format $.xls$ : vous aurez besoin d'installer un package pour pouvoir les lire. La commande pour ce faire est \textit{install.packages("nom du package")}. Installer le package \textit{gdata}.
\item Analyser votre table.
\end{enumerate}

\section{Exercice 5}

\begin{enumerate}
\item Télécharger et installer le package \textit{ISwR} (Introductory Statistics with R).
\item Utiliser la commande \textit{summary} pour analyser rapidement la table \textit{bp.obese}. L'échantillon provient d'un échantillon de population mexicaine en Californie, et la table décrit $3$ variables : le sexe (femme = $1$, homme =$0$), le ratio d'obésité (\textit{obese}) et la pression sanguine systolique en $mm$ de mercure (\textit{bp}).
\item Représenter les données dans un graphe, en utilisant des symboles différents pour les hommes et les femmes.
\item Expliquer la pression sanguine en fonction du ration d'obésité, puis du ratio d'obésité et du sexe.
\item Tracer sur un même graphe les courbes correspondant aux régressions dans les 2 modèles. Soignez la présentation (couleurs différentes, légende,...)
\end{enumerate}

\section{Exercice 6}
\begin{itemize}
\item Télécharger la librairie \textit{MASS}.
\item Analyser rapidement la table \textit{cats} et afficher les variables les unes en fonctions des autres par paires.
\item Effectuer une régression linéaire selon le modèle $Hwt \sim Bwt * Sex $. Cela apporte-t-il quelque-chose par rapport au modèle $Hwt \sim Bwt + Sex $ ? 
\item Visualiser les composantes de votre régression.
\item En extraire les prédiction, les coefficients, les résidus, lés résidus studentisés, et la formule du modèle.
\item Tracer le \textit{qqplot} des résidus studentisés ainsi que la première bissectrice.
\item Tracer le graphe des résidus contre les prédictions.
\item Tracer le graphe des distance de Cook.
\item Observer les attributs que vous donne \textit{summary}
\item Afficher le $R^2$ ajusté de la régression, le nombre de degrés de liberté résiduels, la matrice de variance-covariance des paramètres estimés. 
\end{itemize}

