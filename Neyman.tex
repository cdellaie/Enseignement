\section{Principe de Neyman : décision à $2$ points}

\begin{enumerate}

\item Soit $f$ la densité d'une loi de probabilité sur $\R$, et $\mathcal E$ l'expérience statistique engendré par un $n$-échantillon de loi $p_\theta(x)=f(x-\theta)$. On suppose que $\Theta =\{0, \theta_0\}$ avec $\theta_0\neq 0$. On veut tester $H_0| \theta = 0$ contre $H_1| \theta= \theta_0$.
\begin{enumerate}
\item Décrire l'expérience statistique et donner la vraisemblance du modèle.
\item Donner la zone de rejet du test de Neyman-Pearson de niveau $\alpha$ associé à $H_0$ et $H_1$.
\end{enumerate}

\item L'expérimentateur observe une seule réalisation d'une v.a. $X$ de loi de Poisson de paramètre $\theta>0$. On veut tester $H_0| \theta = \theta_0$ contre $H_1| \theta= \theta_1$, où $\theta_0\neq \theta_1$.
\begin{enumerate}
\item Donner la zone de rejet du test de Neyman-Pearson de niveau $\alpha$ associé.
\item Sachant que $\mathbb P_{\theta_0}(X>9)=0.032$ et $\mathbb P_{\theta_1}(X>8)=0.068$, donner une zone de rejet explicite pour $\alpha = 0.05 = 5\%$. Le test est-il optimal ?
\end{enumerate}

\end{enumerate}

\section{Neyman-Pearson : familles à rapport de vraisemblance monotone}

\begin{enumerate}
\item Soit $\mathcal E$ l'expérience statistique engendrée par un $n$-échantillon de loi normale $\mathcal N (\theta,\sigma^2)$, où $\sigma^2$ est connu, et $\theta\in \Theta =\R$. On souhaite tester $H_0| \theta = \theta_0$ contre $H_1| \theta= \theta_1$, où $\theta_0<\theta_1$.
\begin{enumerate}
\item Décrire le modèle ainsi que la vraisemblance. On choisira la mesure de Lebesgue comme mesure dominante.
\item Calculer le rapport de vraisemblance \[\frac{f(\theta_1,Z)}{f(\theta_0,Z)}.\]
\item Donner la zone de rejet pour le test de Neyman-Pearson associé.
\end{enumerate}

\item Pour la même expérience statistique, on a un test optimal (uniformément plus puissant ) de $H_0$ contre $H_1$ donné par la région de rejet 
\[\mathcal R = \{\overline X_n >c\}\]
où $c$ est solution de $\mathbb P_{\theta_0}(\overline X_n > c)=\alpha$.
\begin{enumerate}
\item Calculer explicitement la valeur de la constante $c=c(\theta_0,\alpha)$.
\item Calculer la puissance de ce test.
\end{enumerate}
\end{enumerate}

\section{Exercice}
L'expérimentateur observe $2$ échantillons indépendants $X_1,...,X_n$ et $Y_1,...,Y_m$ de tailles distinctes $n\neq m$, de lois respectives $\mathcal N(\mu_1,\sigma^2_1)$ et $\mathcal N(\mu_2,\sigma^2_2)$. Il souhaite tester
\[H_0\ : \ \mu_1=\mu_2 \quad \text{contre} \quad H_1  : \ \mu_1\neq \mu_2 .\]
Si $s_{n,1}^2 = \frac{1}{n}\sum_{j=1}^n (X_j-\overline X_n)^2$ et $s_{m,2}^2 = \frac{1}{m}\sum_{j=1}^m (Y_j-\overline Y_m)^2$, construire un test basé sur la statistique 
\[T_{n,m}=\frac{\overline X_n - \overline Y_m}{\sqrt{s_{n,1}^2+s_{m,2}^2}}\]
et étudier sa consistance.


\section{Neyman-Pearson : loi exponentielle}

On observe un $n$-échantillon $\underline x=(X_1,...,X_n)$ de variables iid de loi exponentielle de paramètre $\lambda>0$, de densité 
\[x\mapsto \lambda \exp(-\lambda x)1_{x\leq 0}.\]
\begin{enumerate}
\item Rappeler l'espérance et la variance (les calculer si besoin) d'une loi exponentielle de paramètre $\lambda$. On rappelle que $2\lambda \sum_{j=1}^n X_j$ suit alors une loi du $\chi^2$ à $2n$ degrés de liberté.
\item Ecrire le modèle statistique engendré par l'observation $ \underline x$.
\item Calculer l'estimateur du maximum de vraisemblance $\hat \lambda^{MV}$ de $\lambda$.
\item Montrer que $\hat \lambda^{MV}$ est asymptotiquement normal, et calculer sa variance limite.
\item Soient $0<\lambda_0 < \lambda_1$. Construire un test d'hypothèse de 
\[H_0 : \lambda =\lambda_0 \text{ contre } H_1 : \lambda = \lambda_1\]
de niveau $\alpha$ et uniformément plus puissant. Expliciter le choix du seuil définissant la région critique. Montrer que le test est consistant, i.e. que l'erreur de seconde espèce du test tend vers $0$ lorsque $n\rightarrow \infty$.
\end{enumerate}
