\section{Follicule de rose}

Cet exercice est issu du live \textit{Initiation à la statistique avec R} de F. Bertrand et M. Maumy-Bertrand aux édition Dunod.\\

\begin{enumerate}
\item Télécharger le package \textit{BioStatR} et importer la librairie ainsi que la table \textit{Mesures}.
\item Tracer le nuage de points de la taille en fonction de la masse des follicules de roses.
\item Pouvez-vous soupçonner une relation linéaire entre ces deux variables ?
\item Effectuer la régression linéaire de la taille sur la masse. Tracer la droite de régression sur le graphique.
\item Donnez une prédiction plausible de la taille d'un follicule si sa masse est de $42$ grammes.
\item Comment appellez-vous l'écart entre la valeur observée d'une observation et celle prédit par le modèle ?
\item La droite des moindres carrés passe-t-elle par le point moyen $(\overline x_n, \overline y_n)$ ? Si oui, démontrer-le.
\item Donner le coefficient de détermination $R^2$.
\item Donner la part de variance expliquée et totale.
\item Donnez la valeur de l'estimation de $\sigma^2$.
\item Effectuer les deux tests suivants au risque $0.05\%$ :
\begin{itemize}
\item $H_0 \ | \ \beta_0 = 0$ contre $H_1 \ | \ \beta_0 \neq 0$.
\item $H_0 \ | \ \beta_1 = 0$ contre $H_1 \ | \ \beta_1 \neq 0$.
\end{itemize}
\end{enumerate}
