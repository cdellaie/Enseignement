\section{Algèbre linéaire}

\subsection{Réduction des endomorphismes}

Ce numéro présente la réduction des endomorphisme vue sous l'angle des $k[X]$-modules. Pour cela, nous nous inspirons fortement du livre de R. Mneimé, \textit{Réduction des endomorphismes}\cite{mneimnereduction}. On rappelle que $f,g\in \mathcal L(E)$ sont équivalents ssi il existe $T\in GL(E)$ tel que $Tf=gT$.\\

Soit $k$ un corps, $E$ un $k$-espace vectoriel et $f\in \mathcal L(E)$ un endomorphisme de $E$. La loi
\[\forall P\in k[X],\forall v\in E, \ P.v :=P(f)v\]
définit une structure de $k[X]$-module sur $E$. On notera $E_f$ le $k[X]$-module obtenu. Une remarque importante : soient $f,g\in \mathcal L(E)$, alors une endormorphisme $T\in \mathcal L(E)$ induit un morphisme de $k[X]$-module $\tilde T\in Hom_{k[X]}(E_f,E_g)$ ssi $Tf=gT$. On a alors facilement la proposition suivante :

\begin{prop}
Soient $f,g\in \mathcal L(E)$. Alors $E_f$ et $E_g$ sont isomorphes ssi $f$ et $g$ sont équivalents.
\end{prop}

Rappelons que $k[X]$ est un anneau euclidien. L'algorithme du pivot de Gauss assure donc que pour toute matrice $A\in M_n(k[X])$, il existe $Q_0,Q_1\in GL_n(R)$ et des polynômes $P_1,...,P_r\in k[X]$ vérifiant $P_j|P_{j+1}$ tels que 
\[Q_0AQ_1= \begin{pmatrix} 
P_1 & 0 &     &   &  & \\
0 &... & 0    &   &  & \\
  &   0  & P_r &   &  & \\
  &    &      & 0 &  &  \\
  &    &      &   & ..  &  \\
  &    &      &    &  & 0\\
\end{pmatrix}\]
avec $Q_j$ qui sont produit de transvections. Lorsque $A = \text{mat}(f)-X.id_E$ la famille de polynômes $P_1,...,P_r$ sont appelés les facteurs invariants de $f$.\\

Dans un tel anneau, l'identité $A.Com(A)^T = Com(A)^T . A = det(A)I_n$ a toujours un sens, et fournit une preuve simple du théorème de Cayley-Hamilton. En effet, appliquée à la matrice $A-XI_n\in M_n(k[X])$, on obtient 
\[\chi_A(X) I_n = Com(A-XI_n)^T . (A-XI_n)\]
mais dans $E_A$, on a par définition $X.e_i=Ae_i = \sum a_{ij}e_j$ donc $\chi_A(X).e_i=0$ pour tout $i$, i.e. $\chi_A(A)=0$.\\ 

\subsection{Un rappel sur le déterminant}
On notera $\mathfrak S_n $ le groupe des permutations de $\{1,n\}$ et pour $\sigma\in\mathfrak S_n$, $(-1)^\sigma$ sa signature.\\

Le déterminant d'une matrice $A=(a_{i,j})_{i,j}\in \mathfrak M_n(k)$ est souvent défini comme 
\[det(A) = \sum_{\sigma\in \mathfrak S_n} (-1)^\sigma \prod_{i=1,n} a_{i,\sigma(i)},\]
ce qui rend difficile de montrer (autrement que par calcul) la multiplicativité. Le but de ce numéro est de donner une définition plus conceptuelle qui rend la démonstration presque automatique.\\

\begin{definition}
Soit $V$ un $k$-espace vectoriel de dimension $n\in \N$ et soit $p$ un entier. On définit le $k$-espace vectoriel des formes $p$-linéaires alternées comme :
\[\Lambda_p V = \{ \phi : V^{\times p} \rightarrow \text{ t.q. } \phi(\sigma.x)= (-1)^\sigma x\ ,\forall \sigma\in \mathfrak S_n \text{ et } \phi \text{ est linéaire en chaque variable}\}\]
où :
\begin{itemize}
\item[$\bullet$] $V^{\times p} = V\times ... \times V$ est le produit ($p$ fois) de $V$,
\item[$\bullet$] $\sigma.(x_1,...,x_p) = (x_{\sigma(1)},...,x_{\sigma(p)}) $ est l'action canonique de $\mathfrak S_p$ sur $V^{\times p}$,
\end{itemize}
\end{definition}

On remarque facilement que $dim_k (\Lambda_p V )= \dbinom{n}{p}$, en particulier les formes $n$-linéaire alternées sont toutes proportionnelles.

\begin{definition} Si $V = k^n$ et $\{e_j\}_{j=1,n}$ est la base canonique.
Le déterminant est l'unique forme $n$-linéaire alternée $det$ qui satisfait $det(e_1,...,e_n)=1$. Le déterminant d'une application linéaire est défini comme 
\[det(f) = det(f(e_1),...,f(e_n)).\]
\end{definition}

\begin{prop}
Soient $f,g\in\mathcal L(V)$, alors :
\[det(f \circ g ) = det(f)det(g).\] 
\end{prop}

\begin{dem}
L'application $\phi : (v_1,...,v_n) \mapsto det(f(v_1),...,f(v_n))$ est $n$-linéaire alternée, elle est donc proportionnelle au déterminant. Elle vaut $det(f)$ sur $(e_1,...,e_n)$ donc $\phi(v_1,...,v_n) = det(f) det(v_1,...,v_n)$. On a donc :
\[\phi(g(e_1),...,g(e_n)) = \left\{\begin{array}{l}det(f)det(g) \\ det(f\circ g) \end{array}\right.\]
\qed 
\end{dem}



