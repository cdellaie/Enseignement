%Quelques remarques :
%\begin{itemize}
%\item[$\bullet$] Le TP est à rendre par mail avant le mercredi $3$ janvier. Votre fichier sage doit être nommé $TP3\_Nom$ où vous remplacerez Nom par votre nom de famille. Essayez de commenter votre code, vous ne le faîtes pas assez. Un code mal commenté est un code incompréhensible.
%\item[$\bullet$] Dans l'énoncé et dans tous les suivants, les verbes "démontrer", "montrer", etc. impliquent que l'on attend de vous une démonstration mathématique écrite, tandis que les instructions "implémenter une fonction", "écrire un programme", etc. signifient que vous devez utiliser Sage.
%\end{itemize}

\section{Calcul rapide de puissance}

\begin{enumerate}
\item Implémenter une fonction qui à deux entiers $a$ et $m$ retourne $a^m$.
\item Voici un algorithme dit rapide pour élever un entier à un certaine puissance. Si on écrit $m$ en base binaire, soit $m=\overline{m_k ... m_1 m_0}^{(2)} = \sum_{j=0}^k m_j 2^j$, on peut se servir récursivement de l'identité
\[a^m = a^{m_0 }\big(a^{m_1}(a^{m_2}...)^2\big)^2\]
pour diminuer les coûts de calculs. Implémenter une telle fonction qui utilise moins de $2 E[\log m]$ multiplications, avec $E$ la partie entière.
\end{enumerate}

\section{Tests de primalité}

\subsection{Premier algorithme naif} 
Le premier test de primalité qui vient à l'esprit est de parcourir à l'aide d'une boucle tous les entiers de $2$ à $n-1$ et de vérifier si l'un deux divise $n$. Un instant de réflexion permet de comprendre que l'on peut se limiter aux nombres inférieurs à $\sqrt n$. Pourquoi ?\\ 
Implémenter une fonction qui effectue ce test.

\subsection{Test de Fermat}

Nous allons utiliser le petit théorème de Fermat, que voici.
\begin{thm}
Soient $p$ un nombre premier et $a$ un entier. Alors $a^{p-1} \equiv 1\pmod p$ pour $a$ premier à $p$ et $a^p\equiv a \pmod p$ pour tout entier $a$.
\end{thm}

Si l'on parvient à trouver un entier $a\in \{1,2,...,n-1\}$ tel que $a^{n-1}\neq 1\mod n$ alors $n$ n n'est pas premier. Un tel $a$ est appelé un témoin de Fermat.
\begin{enumerate}
\item Implémenter une fonction $Fermat(a,n)$ qui prend deux entiers $a$ et $n$ en entrée, et qui retourne $True$ si $a$ est une témoin de Fermat pour $n$, $False$ sinon.
\item Implémenter une fonction qui prend en entrée deux entiers $n$ et $M$, qui effectue au plus $M$ fois le test $Fermat(a,n)$ sur un nombre $a$ tiré au hasard entre $2$ et $n-1$, et qui s'arrête dès que $Fermat(a,n)$ renvoie $True$. Cette fonction doit retourner une chaîne de caractère : "n n'est pas premier" si elle a trouvé un témoin de Fermat, et "n est probablement premier" sinon.
\end{enumerate}

\subsection{Deux tests probabilistes}

Importer le package $random$ grâce à la commande $import \ random$. Que retourne l'évaluation de $random.randint(a,b)$, où $a$ et $b$ sont deux entiers ?

\subsubsection{Test de Miller}

\begin{thm} Soit $p>2$ un nombre premier, et $s$ et $t$, $t$ impair, tels que $p-1=2^s t $. Soit $a$ un entier non divisible par $p$. Alors , ou bien $a^t \equiv 1\pmod p$, ou bien il existe un entier $j$ tel que $0\leq j < s $ et $a^{2^j t} \equiv -1\pmod p$.
\end{thm}
De ce théorème, on déduit que si $n$ est un entier impair, alors l'existence d'un entier $a$, $1<a<n$, tel que 
\[a^t\neq 1\pmod n \quad\text{ et }\quad a^{2^j t}\neq -1\pmod n\]
pour $j=0,..,s-1$ assure que $n$ est composé. Un tel $a$ est appelé témoin de Miller pour $n$.(La méthode de ce numéro a été proposé par Gary Miller)
\begin{enumerate}
\item Implémenter une fonction $Miller(a,n)$ qui prend deux entiers $a$ et $n$ en entrée, et qui retourne $True$ si $a$ est une témoin de Miller pour $n$, $False$ sinon.
\item Implémenter une fonction qui prend en entrée deux entiers $n$ et $M$, qui effectue au plus $M$ fois le test $Miller(a,n)$ sur un nombre $a$ tiré au hasard entre $2$ et $n-1$, et qui s'arrête dès que $Miller(a,n)$ renvoie $True$. Cette fonction doit retourner une chaîne de caractère : "n n'est pas premier" si elle a trouvé un témoin de Miller, et "n est probablement premier" sinon.
\end{enumerate}

\subsubsection{Test de Solovay-Strassen}
Soit $n$ et $m$ deux entiers. On dit que $n$ est un résidu quadratique modulo $m$ s'il existe un entier $a$ tel que $n\equiv a^2\pmod m$, i.e. si $n$ est un carré dans l'anneau $\Z/m\Z$. Si $p$ est premier, le symbole de Legendre $\big(\frac{a}{p}\big)$ est un nombre défini comme valant $0$ si $p$ divise $n$, $1$ si $p$ ne divise pas $n$ et $n$ est un résidu quadratique modulo $p$, $-1$ sinon. Un formule dûe à Euler permet de calculer le symbole de Legendre grâce à l'algorithme des puissances rapides du premier numéro.

\begin{prop}[Euler]
Soit $p$ premier impair. L'anneau $\Z/p\Z$ possède $p$ éléments qui sont des carrés : $0$ et $\frac{p-1}{2}$ éléments de $(\Z/p\Z)^\times$. De plus $\big(\frac{a}{p}\big)\equiv a^{\frac{p-1}{2}}\pmod p$.
\end{prop}

Si $n\geq 3$ est impair, on rappelle que $\big(\frac{a}{n}\big)$ est le symbole de Jacobi, défini comme suit. On décompose $n$ en facteurs premiers $n=p_1^{\alpha_1}...p_k^{\alpha_k}$, alors 
\[\big(\frac{a}{n}\big)=\big(\frac{a}{p_1}\big)^{\alpha_1}...\big(\frac{a}{p_k}\big)^{\alpha_k}\]

\begin{thm}[Solovay-Strassen]
Soit $n>2$ un entier impair tel que $\big(\frac{a}{n}\big)\equiv a^{\frac{n-1}{2}}\pmod n$ pour tout entier $a$ premier à $n$. Alors $n$ est premier. 
\end{thm}

On peut en déduire que si $n$ est premier, $\big(\frac{a}{n}\big)\equiv a^{\frac{n-1}{2}}\pmod n$ pour tout entier $a$ premier à $n$, et si $n>2$ est composé, l'ensemble des $a$ premiers à $n$ tels que $0<a<n$ et $\big(\frac{a}{n}\big)\equiv a^{\frac{n-1}{2}}\pmod n$ a au plus $\frac{\varphi(n)}{2}$ éléments.

\begin{enumerate}
\item Implémenter une fonction qui calcule le symbole de Jacobi. (Difficile, même avec le théorème de réciprocité quadratique) Si vous n'y arrivez pas, vous pouvez utiliser la fonction $kronecker(a,n)$ que Sage a déjà en mémoire, et qui calcule le symbole de Jacobi.
\item Comme dans les numéros précédents, implémenter une fonction qui teste si un nombre est un témoin de Solovay-Strassen pour $n$, et ensuite une autre fonction qui effectue ce test aléatoirement au plus $M$ fois.
\end{enumerate}

\subsection{Exercices}

\subsubsection{Indicatrice d'Euler}

On dit qu'une fonction $f: \N\rightarrow \N$ est multiplicative si, lorsque $pgcd(n,m)=1$, alors $f(nm)=f(n)f(m)$. Attention ce n'est pas une définition standard : généralement multiplicatif signifie respecter la mutliplication, et ce de façon inconditionnelle. Un fonction multiplicative est déterminée par ses valeurs sur les puissances de nombres premiers. On note $\varphi(n)$ l'indicatrice d'Euler, i.e. le nombre d'entiers $<n$ et premiers avec $n$.

\begin{enumerate}
\item Montrer que si $f$ est multiplicative, alors 
\[g(n)=\sum_{d|n} f(d)\]
l'est aussi.
\item Montrer que \[n=\sum_{d|n}\varphi(d).\]
\end{enumerate} 

\subsubsection{}

\begin{enumerate}
\item Démontrer le petit théorème de Fermat.
\item Coin de la culture : chercher le théorème de réciprocité quadratique si vous ne le connaissez pas.
\end{enumerate}
