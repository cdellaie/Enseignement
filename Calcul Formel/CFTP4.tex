\section{Code de Hamming}

\begin{enumerate}
\item Implémenter une fonction \textbf{Test} qui prend en entrée un message bruité ainsi qu'une matrice définissant un code BCH; et retourne $True$ si le message appartient au code défini par $H$, $False$ sinon.
\item Implémenter une fonction \textbf{Code} qui prend en entrée un message à envoyer ainsi qu'une matrice définissant un code BCH; et retourne le message codé.
\item Implémenter une fonction qui prend en entrée un message non bruité et le bruite. On prendra pour cela un bit du message au hasard que l'on inverse ($0\mapsto 1$ et $1\mapsto 0$).
%Par exemple, on peut mettre une probabilité faible sur chaque bit d'être inversé. On pourra aussi choisir un bit au hasard, et l'inverser.
\item Implémenter une fonction qui, étant donné un code et un message reçu, vérifie s'il y a une erreur, et le cas échéant, corrige l'erreur. La fonction doit retourner le message corrigé. En cas d'erreur, en plus du message corrigé, la fonction afficher un message du type "Erreur detectée sur le bit numéro j" avec $j$ la position de l'erreur.
\item Combien le code de Hamming $H(4,3)$ contient-t-il de mots ? Calculer sa distance, sa capacité de détection et de correction.
\end{enumerate}

\section{Construction du polynôme générateur}

\fbox{\begin{minipage}{0.9\textwidth} \textbf{Polynôme cyclotomique}\\
\textbf{Entrée :} $n\in \N^*$\\
\textbf{Sortie :} $\Phi_n$
\begin{enumerate}
\item Déterminer la décomposition en facteur premier de $n= p_1^{\alpha_1}...p_k^{\alpha_k}$
\item $m=p_1 p_2... p_k$
\item $P_0=X-1$
\item Pour $j=1,..,k$ faire $P_j \leftarrow P_{j-1}(X^{p_j})// P_{j-1}(X)$
\item Retourner $P_k{X^{n/m}}$
\end{enumerate}
\end{minipage}}\\
\\

\fbox{\begin{minipage}{0.9\textwidth} \textbf{Cantor-Zassenhaus}\\
\textbf{Entrée :} $P\in \mathbb F_2[X] $ unitaire sans facteur carré ayant tous des facteurs irréductibles de même degré $d$, $deg\ P=n>0$. \\
\textbf{Sortie :} Un facteur non trivial de $P$, ou $False$.
\begin{enumerate}
\item Choisir $Q\in \mathbb F_2[X]$ au hasard de degré $<n$.
\item $g=pgcd(P,Q)$
\item Si $g=1$, alors retourner $g$
\item $a=T_d(Q) \pmod P$
\item $g\leftarrow pgcd(a-1,P)$
\item Si $g\neq 1$ et $g\neq P$, retourner $g$, sinon retouner $False$
\end{enumerate}
\end{minipage}}\\
\\