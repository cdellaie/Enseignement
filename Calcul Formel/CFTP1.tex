Quelques adresses utiles :
\begin{itemize}
\item[$\bullet$]le site de l'agrégation de mathématiques \url{http://agreg.org}, vous y trouverez des textes pour vous entraîner, et surtout les comptes rendus du jury. Aussi, la liste des logiciels acceptés à l'agreg : \textbf{Python, Scilab, Octave, Sage, Maxima, Xcas, R}. Tous sont libres et gratuits.
\item[$\bullet$]Nous allons travailler avec Sage, que vous pouvez télécharger sur la page \url{http://www.sagemath.org/fr/}. Vous pouvez aussi travailler directement dans une page ouverte dans le navigateur.
\end{itemize}

\section{Prise en main de SAGE}

\begin{enumerate}
\item Ouvrir la page \url{https://cloud.sagemath.com/} dans votre navigateur, créer un compte gratuit et un nouveau fichier (\textit{New sageworksheet}) que vous nommerez \text{TP1} par exemple.
\item Trouver et télécharger le tutoriel SAGE : vous pouvez le faire chez vous pour vous entraîner.
\item Pour évaluer les instructions contenues dans une cellule, vous pouvez soit cliquer sur le lien \textit{evaluate} ou \textit{Run} en haut à gauche, soit utiliser le raccourci clavier MAJ+ENTREE. Saisir les expressions suivantes et les évaluer :
\[9+3,\ 6*5,\ 63//9,\ 17//5,\ 17\%5.\]
\item Que donne l'évaluation des instructions suivantes ?
\[cos?,\ RationalField? \ , PolynomilaRing? \ .\] 
\item Pour affecter une valeur à une variable, on utilise le signe $=$. Affecter la valeur $cos(\frac{\pi}{2})$ à une variable $x$, puis évaluer $x$, $print(x)$. Pour un affichage graphique, utilisez $x.show()$. Que renvoie l'évaluation de $x.n(digits=10)$ ?
\item SAGE permet de faire du calcul symbolique. Pour déclarer des variables $x$ et $y$, on utilise la commande $var('x,y')$. Evaluer
\[\begin{array}{l}
var\ ('x,y') \\
z = cos (x )^2 + sin (y )^2 \\
print\ z.\ subs\_expr\ (x== pi /2)\\
print \ z(y=pi /2)
\end{array}\]
Si SAGE indique une erreur (\textit{deprecation}), c'est seulement que vous avez déjà utilisé un des noms de variables auparavant. Pour corriger ce problème, il suffit de réinitialiser les variables en évaluant $reset()$ au début de votre code.
\item Définir une fonction de la façon suivante :
\[\begin{array}{l}
reset ()\\
var\ ('x,y') \\
f(x,y) \ = \ x/ sin (x) + sqrt (y) \\
f
\end{array}\]
La fonction $f$ est désormais un objet, évaluer $f.parent()$ pour savoir lequel. Evaluer \[f. show (), \  f. limit (x =0). show (),\ f. diff (x). show ().\]
\item On peut aussi utiliser la commande $def$ pour définir une fonction : 
\[\begin{array}{l}
def\ FONCTION(entree):\\
instructions\\
return(sortie)
\end{array}
\]
Définir une fonction qui prend en entrée un nombre réel $x$ et renvoie son carré $x^2$.
\section{Arithmétique}
\item Créer une fonction qui prend en entrée deux entiers positifs, et renvoie leur \textit{pgcd}, grâce à l'algorithme d'Euclide.
\item Créer une fonction qui prend en entrée deux entiers positifs $x$ et $y$, et renvoie leur \textit{pgcd} $d$ ainsi que deux entiers $u$ et $v$ qui vérifient $ux+vy=d$. Vous utiliserez l'algorithme d'Euclide étendu.
\item Créer un fonction qui prend en entrée deux listes d'entiers $[a_1,...,a_k]$ et $[n_1,...,n_k]$ et renvoie une solution du système de congruences
\[\left\{\begin{array}{rcl}
 x & = & a_1 \ \text{mod }n_1\\
   &  ... & \\
 x & = & a_k \ \text{mod }n_k\\
\end{array}\right.\]
\item SAGE permet de faire de l'arithmétique avec des polynômes. Construire l'anneau des polynômes à coefficients rationels à une indeterminée en évaluant $R=QQ['x']$. 
\item Spécifier le nom de la variable grâce à $T=R.gen()$.
\item Créer une fonction qui effectue l'algorithme d'Euclide étendu sur deux polynômes, et le tester sur $P=T^2+1$ et $Q=T^3+2*T^2+1$.

\end{enumerate}