Quelques adresses utiles :
\begin{itemize}
\item[$\bullet$]le site de l'agrégation de mathématiques \url{http://agreg.org}, vous y trouverez des textes pour vous entraîner, et surtout les comptes rendus du jury. Aussi, la liste des logiciels acceptés à l'agreg : \textbf{Python, Scilab, Octave, Sage, Maxima, Xcas, R}. Tous sont libres et gratuits.
\item[$\bullet$]Nous allons travailler avec Sage, que vous pouvez télécharger sur la page \url{http://www.sagemath.org/fr/}. Vous pouvez aussi travailler directement dans une page ouverte dans le navigateur.
\end{itemize}

\section{Prise en main de SAGE}

\begin{enumerate}
\item Ouvrir la page \url{https://cloud.sagemath.com/} dans votre navigateur, créer un compte gratuit et un nouveau fichier (\textit{sageworksheet}) que vous nommerez \text{TP1} par exemple.
\item Trouver comment ouvrir l'aide.
\item Créer une fonction qui prend en entrée deux entiers positifs, et renvoie leur \textit{pgcd}, grâce à l'algorithme d'Euclide.
\item Créer une fonction qui prend en entrée deux entiers positifs $x$ et $y$, et renvoie leur \textit{pgcd} $d$ ainsi que deux entiers $u$ et $v$ qui vérifient $ux+vy=d$. Vous utiliserez l'algorithme d'Euclide étendu.

\end{enumerate}