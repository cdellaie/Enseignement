\section{Complexité}

Au premier TP, vous avez implémenté les algorithmes d'Euclide et d'Euclide étendu. Ces algorithmes se basent sur la propriété $pgcd(x,y)=pgcd(y,r)$ où $r$ est le reste de la division euclidienne de $x$ par $y$. La terminaison de l'algorithme fournit entre autre une preuve de l'identité de Bézout : il existe des entiers $u$ et $v$ tels que $ux+vy=pgcd(x,y)$. Nous allons nous intéresser au coût de ces algorithmes.

\begin{enumerate}
\item Soient $x$ et $y$ deux entiers. Montrer que $x$ et $y$ sont premiers entre eux ssi il existe $u$ et $v$ dans $\Z$ tels que $ux+vy=1$.\\ 

La suite de Fibonacci est définie par 
$\left\{\begin{array}{c} F_0 = 0, F_1 = 1\\
F_{n+1}=F_n+F_{n-1}\end{array}\right.$

\item Calculer les $7$ premiers termes de la suite.
\item Montrer que, si $n>0$, alors 
\[\begin{pmatrix}1 & 1 \\ 1 & 0\end{pmatrix}^n = \begin{pmatrix}F_{n+1} & F_n \\ F_n & F_{n-1}\end{pmatrix}\]
En déduire $F_{n+1}F_{n-1}-F_n^2 = (-1)^n$.
\item Montrer que $F_{n+m} = F_{n+1}F_m+F_n F_{m-1}$. En déduire : $pgcd(F_{n+m},F_m)= pgcd(F_m,F_n)$, puis $pgcd(F_n,F_m)=F_{pgcd(n,m)}$.
\item Soit $\varphi = \frac{\sqrt{5}+1}{2}$. Montrer que 
$F_n= \frac{1}{\sqrt{5}}(\varphi^n-(-\varphi)^{-n}),$
et en déduire que $F_n$ est l'entier le plus proche de $\frac{\varphi^n}{\sqrt 5}$.
\item Montrer la proposition suivante, que l'on doit à Lamé (1845). \\
Soient $x$ et $y$ deux entiers tels que $0<y<x$ et soit $d$ leur $pgcd$. Si l'algorithme d'Euclide partant de $(x,y)$ s'arrête au bout de $n$ pas, on a 
\[x \geq d F_{n+2}, \quad y \geq d F_{n+1}.\]
Est-ce optimal ?
\item Montrer que $n\leq \frac{3}{2} log (F_{n+1})+1$. En déduire une majoration du nombre de pas de l'algorithme d'Euclide.\\

Pour estimer le coût de l'algorithme d'Euclide, nous allons d'abord définir le modèle dans lequel nous nous plaçons, i.e. établir le coût de chaque opération arithmétique. \\

$\bullet$ Le coût d'une addition ou soustraction $m\pm n$ est majorée par $c_+ \max(\log|m|,\log |n|)$.\\
$\bullet$ Le coût d'une multiplication $m\times n$ est majorée par $c_\times \log |m| \log |n|$.\\
$\bullet$ Le coût d'une division euclidienne de $m$ par $n$, où $0<n\leq m$, est majorée par $c_\% log|m|\log |m/n|$.\\

Ce modèle s'appelle le modèle à coûts bilinéaires. Montrer que dans ce modèle, le coût du pgcd de $x$ et $y$ avec $0\leq x< y$ est majoré par $c_\% (\log y )^2$.
\end{enumerate}

\section{Cryptographie}

On s'intéresse dans cette section à deux systèmes de cryptographie "à clefs publiques". Ces deux méthodes permettent à deux individus, que l'on appellera comme le veut l'usage Alice ($A$) et Bob ($B$), d'échanger des messages sans qu'une tierce personne puisse en bénéficier. L'originalité des méthodes à "clefs publiques" tient dans le fait que, plutôt que d'utiliser un code secret, $A$ et $B$ utilisent chacun un "code" connu de tous. Nous étudierons le système de Rabin (1979), ainsi que RSA (1977), du nom de ses auteurs Rivest, Shamir et Adleman. L'originalité du protocole de Rabin est qu'il est "prouvé" : casser le code est démontré être équivalent à savoir factoriser la clé publique, qui est un grand entier, en facteurs premiers. Ce n'est pas le cas de RSA, où l'on sait seulement que si l'on sait factoriser de grands entiers en temps polynômial, alors on peut casser RSA.\\

Le principe de ces méthodes repose sur le choix d'une fonction bijective publique pour chaque personne, dont l'inverse est difficile à calculer. Si l'ensemble des messages est $\mathfrak M$, $A$ choisit une bijection $\phi_A : \mathfrak M \rightarrow \mathfrak M$ et $B$ fait de même. Ces bijections sont publiques : tout le monde peut les trouver dans un "annuaire", et faciles à calculer. Par contre, l'inverse $\phi_A^{-1}$ est difficile à calculer (par exemple nécessite une temps de calcul exponentiel), sauf pour $A$. Si $A$ veut envoyer un message $m\in \mathfrak M$ à $B$, $A$ cherche $\phi_B$ dans l'annuaire, et envoie $s=\phi_B\circ \phi_A^{-1}(m)$ à $B$. Pour le décoder, $B$ cherche $\phi_A$ dans l'annuaire et calcule $\phi_A\circ\phi_B^{-1}(s)=m$. On voit ici le double avantage de cette méthode :  une personne tierce doit en théorie calculer $\phi_B^{-1}$ pour décoder "illégalement" le message, ce qui supposé impraticable, et $B$ est sûr que $A$ lui a envoyé le message. En effet, celui-ci est codé avec $\phi_A^{-1}$ qui est le secret de $A$ ! Les protocoles de Rabin et RSA diffèrent quant à leur choix dans les fonctions de codages.\\

Nous nous contenterons de vouloir transmettre des messages binaires, c'est-à-dire des suites de $0$ et de $1$. On les convertira en base $10$, et l'ensemble des messages sera $\mathfrak M = \Z/N\Z$, où $N$ est un entier assez grand pour que tous vos messages soient plus petits. 
\begin{enumerate}
\item Si tous les messages que l'on veut envoyer contiennent moins de $p$ bits, quels $N$ pouvez-vous choisir ?
\item Créer une fonction qui, à une suite de $0$ et de $1$, retourne sa valeur en base $10$, i.e. $\overline{a_k ... a_0}\mapsto \sum a_j 2^j$.
\end{enumerate}

\subsection{RSA}
Voici le déroulement du protocole RSA.\\

$\bullet$\textbf{Génération des clés}\\
$A$ choisit $2$ grands entiers premiers $p$ et $q$, et calcule $N_A=pq$. $A$ choisit ensuite un entier de taille moyenne $d_A$ premier avec $\varphi(N)=(p-1)(q-1)$. La clé publique de $A$ est $(N_A,d_A)$, sa clé privée $(p,q)$.  \\

$\bullet$\textbf{Fonction de codage}\\
La fonction de codage correspond à $\phi_A(m)= m^d \ mod\ N$. L'inverse de $\phi_A$ est donnée par $\phi_A^{-1}(s)=s^u$ où $u$ est un inverse de $d$ modulo $\varphi(N)$.\\

\begin{enumerate}
\item Montrer que ces fonctions sont bien inverses l'une de l'autre.
\item Y-a-t-il une contrainte sur le message $m$ ? Est-ce grave ? Indication : Calculer la proportion d'entiers non premiers à $N$ si $p$ et $q$ sont très grands, par exemple $\geq 10^50$.
\item Montrer que si l'on ne connait seulement que la clé publique $(N_A,d_A)$, alors calculer l'inverse de $\phi_A$ équivaut à la connaissance de $\varphi(N)$. Conclure.
\item Implémenter une fonction $code$ qui, étant donnée une clé publique $(N_A,d_A)$ et un message $m$, retourne $\phi_A(m)$.  
\item Implémenter une fonction $decode$ qui, étant donnée une clé privée $(p,q,d_A)$ et un message codé $s$, retourne $\phi_A^{-1}(m)$.  
\item Implémenter une fonction $RSA$ qui, étant donnée une clé privée $(p_B,q_B,d_B)$, une clé publique $(N_A,d_A)$ et un message $m$, retourne le message secret qu'envoie $B$ à $A$. 
\end{enumerate} 
 
\subsection{Cryptosystème de Rabin}

Alice et Bob veulent échanger des messages cryptés. Ils utilisent pour cela le crytposystème à clé publique de Rabin. Voici les $3$ étapes de ce protocole.\\

$\bullet$\textbf{Génération des clés}\\
$A$ choisit $2$ grands entiers premiers $p$ et $q$, et calcule $N_A=pq$. La clé publique de $A$ est $N_A$, sa clé privée $(p,q)$.  \\

$\bullet$\textbf{Chiffrement}\\
$B$ souhaite envoyer un message crypté à $A$. Il récupère sa clé publique $N_A$, représente le message comme un entier $m$ entre $0$ et $N_A-1$, et envoie $m^2 $ modulo $N_A$ à $A$.\\

$\bullet$\textbf{Déchiffrement}\\
$A$ reçoit $c$, et souhaite calculer une racine carrée. $A$ calcule donc les racines carrées  $\pm r_p$ de $c$ modulo $p$, et  $\pm r_q$ de $c$ modulo $q$. Le théorème chinois donne alors les racines de $c$ modulo $N_A$ : $m_1$, $N_A-m_1$, $m_2$, $N_A-m_2$. A priori, $A$ ne peut décider lequel de ces $4$ messages est celui que $B$ veut transmettre. Toutefois, ce problème peut être corrigé par redondance : $B$ peut copier les $6$ derniers bits de son message à la fin d'icelui, et $A$ choisit le message qui présente une redondance sur ses $6$ derniers bits.\\

Pour calculer une racine carrée de $c$ modulo $p$ ou $q$, nous utiliserons l'algorithme de Tonelli-Shanks.\\

\fbox{\begin{minipage}{0.9\textwidth} \textbf{Tonelli-Shanks}\\
\textbf{Entrées :} $p$ nombre premier impair, $a$ carré modulo $p$.\\
\textbf{Sortie :} une racine carrée de $a$ modulo $p$.
\begin{enumerate}
\item Trouver $b$ entre $2$ et $p-1$ qui ne soit pas carré modulo $p$.
\item Calculer $s\in \N$ et $t$ impair tels que $p-1=2^s t$.
\item $z \leftarrow b^t$ 
\item $m\leftarrow 0$
\item Pour $j$ de $0$ à $s-1$ faire :\\
 si $(a^t z^m)^{2^{s-1-j}} = -1\ (mod \ p)$ alors $m\leftarrow m+2^j$ fin si.\\
fin pour
\item Retourner $a^{\frac{t+1}{2}}z^{\frac{m}{2}}$.
\end{enumerate}
\end{minipage}}\\
\\

\begin{enumerate}
\item Implémenter l'algorithme de Tonelli-Shanks.
\item Implémenter des fonctions qui génère des clés, chiffre et déchiffre un message dans le protocole de Rabin.
\end{enumerate}

\section{Conclusion provisoire}

Au terme de ce TP, l'élève attentif se posera les questions suivantes : comment fabriquer de très grands nombres premiers ? Comment choisir $d$ ? Quelles sont les méthodes pour factoriser un entier ? Comment choisir $p$ et $q$ de façon à ce que RSA résiste aux méthodes de factorisation ?\\

Pour la première question, nous verrons qu'il existe des tests de primalité qui décident si un entier est premier ou non. Les autres questions sont plus difficiles. \textbf{Coin de la culture :} Depuis les années $80$, les physiciens savent manipuler et observer des objets quantiques tels que les photons, les atomes, les ions, etc. Ce progrès technique a permis l'essor de l'informatique quantique, et la création d'algorithmes basés sur la manipulation de bits quantiques. Peter Shor a proposé un algorithme (l'algorithme de Shor..) qui permet de factoriser les grands entiers en temps polynomial, et permet donc de casser RSA. En pratique, la construction d'ordinateurs quantiques est très ardue (dûe à des problèmes liés à ce que les physiciens appellent la décohérence) et ces ordinateurs ne peuvent manipuler qu'un nombre très restreints de bits. A titre d'exemple, une équipe d'IBM a réussi à factoriser en $2001$ le nombre $15$ en $ 3 \times 5$ grâce à un ordinateur quantique manipulant $7$ qubits. Pour plus de détails, vous pouvez consulter le livre (disponible à la Bibliothèque Universitaire) de Michel Le Bellac, \textit{Introduction à l'information quantique.}
 