
\documentclass[a4paper]{article}

\usepackage[frenchb]{babel}
\usepackage{amsfonts}
\usepackage{amsmath}
%\usepackage[T1]{fontenc}
\usepackage[utf8]{inputenc}
\usepackage{amsthm}
\usepackage{graphicx}
\usepackage{tikz}
\usepackage{hyperref}
\usepackage{amssymb}
\usepackage{tikz-cd}

\usepackage{listings}
\lstdefinestyle{Scilab}{%
  language=Scilab,%
  breaklines=true,%
  frame=l,%
  xleftmargin=\parindent,%
  basicstyle=\ttfamily\small,%
  keywordstyle=\bfseries\color{green!40!black!80},%
  showstringspaces=false,%
  commentstyle=\itshape\color{purple!70},%
  identifierstyle=\color{blue!80},%
  stringstyle=\color{red!80},%
  directivestyle=\color{orange!90!black!80},%
  % otherkeywords={},%
  escapeinside={<latex>}{</latex>},%
}
\lstset{style=Scilab}

\hypersetup{                    % parametrage des hyperliens
    colorlinks=true,                % colorise les liens
    breaklinks=true,                % permet les retours à la ligne pour les liens trop longs
    urlcolor= blue,                 % couleur des hyperliens
    linkcolor= blue,                % couleur des liens internes aux documents (index, figures, tableaux, equations,...)
    citecolor= cyan               % couleur des liens vers les references bibliographiques
    }

\theoremstyle{definition}
\newtheorem{definition}{Définition}
\newtheorem{thm}{Théorème}
\newtheorem{ex}{Exercice}
\newtheorem{lem}{Lemme}
\newtheorem{dem}{Preuve}
\newtheorem{prop}{Proposition}
\newtheorem{cor}{Corollaire}
\newtheorem{conj}{Conjecture}
\newtheorem{Res}{Résultat}

\newcommand{\N}{\mathbb N}
\newcommand{\Z}{\mathbb Z}
\newcommand{\R}{\mathbb R}
\newcommand{\C}{\mathbb C}
\newcommand{\Hil}{\mathcal H}
\newcommand{\Mn}{\mathcal M _n (\mathbb C)}
\newcommand{\K}{\mathbb K}
\newcommand{\B}{\mathbb B}
\newcommand{\Cat}{\mathbb B / \mathbb K}

\setlength\parindent{0pt}


\title{Exercices de Statistiques  \\ Université de Lorraine \\ ~ \\
\textbf{Estimation et théorie des tests}}

\date{} %la date
\author{ Clément Dell'Aiera }


\begin{document}  
\maketitle

\section{Généralités sur l'estimateur du maximum de vraisemblance}
\begin{enumerate}
\item Rappeler les propriétés de l'EMV.
\item Soient $X_j$ des variables exponentielles indépendantes de paramètre $\theta>0$, non-observées, et $T$ un instant de censure. Soit $\mathcal E^n$ l'expérience engendrée par l'observation du $n$-échantillon $X_j^*=\min{\{T,X_j\}}$. Donner une mesure qui domine le modèle et calculer sa vraisemblance.
\item Montrer que l'estimateur du maximum de vraisemblance ne dépend pas du choix de la mesure dominante.  
\end{enumerate}
\section{Exemples de calculs de maximum de vraisemblance}

Pour chaque loi, on considère un $n$-échantillon tiré de façon i.i.d selon cette loi. Proposer un espace des paramètres donnant un modèle identifiable. Donner une mesure dominante si possible. Calculer la vraisemblance du modèle, ainsi que la log-vraisemblance, donner les équations de vraisemblance et déterminer , s'il existe, un estimateur du maximum de vraisemblance.\\

\begin{enumerate}
\item Modèle gaussien standard, de densité par rapport à la mesure de Lebesgue \[f_\theta(x)=\frac{1}{\sqrt{2\pi \sigma^2}} \exp{-\frac{1}{2\sigma^2}}(x-\mu)^2\quad, \theta=(\mu,\sigma).\]
\item Modèle de Bernoulli \[\mathbb P_\theta(X=1)=1-\mathbb P(X=0)=\theta.\]
\item Modèle de Laplace, où $\sigma>0$ est connu, de densité par rapport à la mesure de Lebesgue \[f_\theta(x)=\frac{1}{2\sigma}\exp{(-\frac{|x-\theta|}{\sigma})}.\]
\item Modèle uniforme, de densité par rapport à la mesure de Lebesgue \[f_\theta(x)=\frac{1}{\theta}1_{[0,\theta]}(x).\]
\item Modèle de Cauchy, de densité par rapport à la mesure de Lebesgue \[f_\theta(x)=\frac{1}{\pi(1-(x-\theta)^2)}.\]
\item Modèle de translation. On considère la densité \[h(x)=\frac{ e^{-\frac{|x|}{2}}}{2\sqrt{2\pi|x|}}.\] Le modèle de translation par rapport à la densité $h$ est le modèle dominé par la mesure de Lebesgue sur $\R$ de densités 
\[f_\theta(x)=h(x-\theta)\quad, x\in R,\theta \in \R.\]
\end{enumerate}

\section{Méthode des moments}
\begin{enumerate}
\item Calculer des estimateurs des moments d'ordre $1$ et $2$ pour l'expérience engendrée par l'observation d'un $n$-échantillon de variables exponentielles de paramètre $\theta>0$. Donner l'asymptotique des ces deux estimateurs.
\item On considère le modèle de translation associé à la famille des lois de Cauchy :
\[f_\theta(x)=\frac{1}{\pi(1+(x-\theta)^2)}\quad, x\in\R.\]
On note $g$ la fonction signe, qui vaut $1$ si $x>0$, $-1$. Trouver un estimateur pour $\theta\mapsto \mathbb E[g(X_1)]$ et donner ses propriétés.
\end{enumerate}
\section{Estimation de la fonction de répartition}

On se donne un $n$-échantillon $X_1$,..., $X_n$ i.i.d suivant une loi donnée par la même fonction de répartition (f.d.r) $F$ sur $\R$. $\mathcal F$ dénote l'ensemble des fonctions de répartition sur $\R$.

\begin{enumerate}
\item Décrire l'expérience statistique.
\item Le modèle est-il dominé ?
\item On veut estimer $F(x)=\mathbb P(X\leq x)$.
	\begin{enumerate}
	\item On pose $\hat F_n(x)=\frac{1}{n}\sum_{i=1}^n 1_{X_i\leq x}$. Calculer $\mathbb E[\hat F_n(x)]$ et $V[\hat F_n(x)]$.
	\item Montrer que $\hat F_n(x)$ converge presque-sûrement vers $F(x)$.
	\item Montrer que, si $l(x,y)=(x-y)^2$ est la perte quadratique, \[\sup_{F\in \mathcal F} \mathbb E[l(\hat F_n(x),F(x))]=\frac{1}{4n}.\]
	\item En déduire que $\hat F_n(x)$ converge uniformément en norme $\mathcal L^2$ vers $F(x)$, et donc en probabilité.
	\end{enumerate}
\item \begin{enumerate}
	\item Montrer que \[\mathbb P(|\hat F_n(x) - F(x)|>t)\leq \frac{1}{t^2}Var[\hat F_n (x)]\leq \frac{1}{4nt^2}\]
	\item Soit $\alpha\in ]0;1[$. Déterminer 
	\[t_{\alpha,n}=\inf \{t>0 \ : \ \frac{1}{4nt^2}\leq \alpha\}\]
	et en déduire un intervalle de confiance pour $F(x)$ de niveau $1-\alpha$.
	\item Comment interpréter $I_{n,\alpha}$ ? Quelle est sa précision ?
	\end{enumerate}
\item On pose $\xi_n = \sqrt{n}\frac{\hat F_n(x) - F(x)}{\sqrt{\hat F_n(x)(1-\hat F_n(x))}}$.
	\begin{enumerate}
	\item Déterminer la limite en loi de $\xi_n$.
	\item On note $J_{n,\alpha}$ l'intervalle $[-\phi^{-1}(1-\frac{\alpha}{2});\phi^{-1}(1-\frac{\alpha}{2})]$.Calculer la limite de $\mathbb P(\xi_n\in J_{n,\alpha})$ lorque $n$ tend vers $\infty$.
	\item Donner un intervalle de confiance asymptotique pour $J_{n,\alpha}$, ainsi que sa précision asymptotique.
	\end{enumerate}
\item Soient $Y_j$ des variables aléatoires réelles indépendantes centrées : $\mathbb E Y_j = 0 $ et bornées : $a_j \leq Y_j \leq b_j$. On veut démontrer ce que l'on appelle l'\textit{inégalité de Hoeffding} : pour tout $t>0$, 
\[\mathbb P(\sum Y_j <t )\leq e^{-st}\prod e^{\frac{s^2(b_j-a_j)^2}{8}}\quad\forall s >0.\]
On pose $\Phi_Y(s)=\log \mathbb E[e^{s(Y-\mathbb E Y)}]$. 
	\begin{enumerate}
	\item Montrer que \[\Phi''_Y(s)=e^{-\Phi_Y(s)}\mathbb E[Y^2 e^{sY}]-e^{-2\Phi_Y(s)}\mathbb (\mathbb E[Ye^{sY}])^2.\]
	\item On définit une nouvelle mesure de probabilité par $\mathbb Q(A)= e^{-\Phi_Y(s)}\mathbb E[e^{sY}1_A]$ pour tout borélien $A$. Comment interpréter $ \Phi''_Y(s)$ dans ce cadre ?
	\item Montrer alors que $\Phi_Y(s)\leq s^2 \frac{(b-a)^2}{8}$.
	\item En déduire l'inégalité de Hoeffding.
	\end{enumerate}

\item \begin{enumerate}
	\item Soient $X_j$ des v.a. de Bernoulli de paramètre $p$ et $\overline X_n = \frac{1}{n}\sum_{j=1}^n X_j$, montrer que 
		\[\mathbb P(|\overline X_n-p|>t)\leq 2e^{-2nt^2}.\] 
	\item En déduire un intervalle de confiance de niveau $1-\alpha$ pour $F(x)$.
	\end{enumerate}
\item Comparer les différents intervalles de confiance que vous avez obtenu.
\end{enumerate}


\end{document}




























